\listofnomenclature{lll} % Include a list of Symbols (a three column table)
{
% Symbol & Name & Unit \\
$A$ & plocha štruktúry MOS & $m^2$ \\
$C$ & kapacita & $F$ \\
$C_{i}$ & kapacita napäťovo-nezávislého kondenzátora Q-C metódy & $F$ \\
$C_{iHF}$ & HF kapacita napäťovo-nezávislého kondenzátora Q-C metódy & $F$ \\
$C_{iLF}$ & LF kapacita napäťovo-nezávislého kondenzátora Q-C metódy & $F$ \\
$C_{m}$ & LF kapacita sériovo-paralelného zapojenia Q-C metódy & $F$ \\
$C_{mos}$ & diferenciálna kapacita štruktúry MOS & $F$ \\
$C_{mos}^{HF}$ & vysokofrekvenčná kapacita štruktúry MOS & $F$ \\
$C_{mos}^{LF}$ & nízkofrekvenčná kapacita štruktúry MOS & $F$ \\
$C_{mos}^{TLF}$ & teoretická nízkofrekvenčná kapacita štruktúry MOS & $F$ \\
$C_{ox}$ & kapacita oxidovej vrstvy štruktúry MOS & $F$ \\
$C_{sc}$ & kapacita oblasti priestorového náboja & $F$ \\
$C_{w}$ & parazitná kapacita Q-C metódy & $F$ \\
$C_{x}$ & parazitná kapacita Q-C metódy & $F$ \\
$D$ & dávka implantovaných atómov v polovodiči & $m^{-2}$ \\
$D_{i}$ & dávka implantovaných atómov zadaná v procese implantácie & $m^{-2}$ \\
$D_{it}$ & hustota pascí rozhrania $Si-SiO_2$ & $m^{-2}eV^{-1}$ \\
$D_{n}$ & difúzny koeficient elektrónov & $m^{-2}s^{-1}$ \\
$E$ & energia & $eV$ \\
$E_{c}$ & energia dolného okraja vodivostného pásma & $eV$ \\
$E_{f}$ & energia Fermiho hladiny v polovodiči & $eV$ \\
$E_{i}$ & energia intrinzickej Fermiho hladiny v polovodiči & $eV$ \\
$E_{v}$ & energia horného okraja valenčného pásma & $eV$ \\
$G_{m}$ & vodivosť sériovo-paralelného zapojenia kondenzátorov Q-C metódy & $F$ \\
$h_{ox}$ & hrúbka oxidovej vrstvy štruktúry MOS & $m$ \\
$I,i$ & elektrický prúd & $A$ \\
$I_{g}$ & generačný prúd minoritných nosičov náboja & $A$ \\
$L_{D}$ & Debayova dľžka & $m$ \\
$L_{DE}$ & extrinzická Debayova dľžka & $m$ \\
$N$ & koncentrácia dotojúcich prímesí v polovodiči & $m^{-3}$ \\
$n$ & koncentrácia elektrónov v polovodiči & $m^{-3}$ \\
$N_{A}$ & koncentrácia akceptorov & $m^{-3}$ \\
$N_{b}$ & koncentrácia substrátu & $m^{-3}$ \\
$N_{D}$ & koncentrácia donorov & $m^{-3}$ \\
$N_{max}$ & maximálna koncentrácia dotujúcich prímesí v polovodiči & $m^{-3}$ \\
$P$ & koncentrácia dier v polovodiči & $m^{-3}$ \\
$Q$ & elektrický náboj & $C$ \\
$Q_{dc}$ & poruchový náboj v $SiO_2$ a na rozhraní s polovodičom a kovom & $C$ \\
$R_{p}$ & stredná hodnota rozloženia implantovaných atómov v polovodiči & $m$ \\
$\Delta R_{p}$ & rozptyl rozloženia implantovaných atómov v polovodiči & $m$ \\
$T$ & teplota & $K$ \\
$t$ & čas & $s$ \\
$u$ & normovaný elektrický potenciál v polovodiči & \\
$u_f$ & normovaný Fermiho potenciál v polovodiči & \\
$u_s$ & normovaný potenciál na povrchu polovodiča & \\
$V$ & napätie & $V$ \\
$V_a$ & napätie na sériovo-paralelnom zapojení kondenzátorov Q-C metódy & $V$ \\
$V_{fb}$ & napätie vyrovnaných pásov štruktúry MOS & $V$ \\
$V_{g}$ & napätie na hradlovej elektróde štruktúry MOS & $V$ \\
$V_i$ & napätie v spoločnom bode sériovo-paralelného zapojenia Q-C metódy & $V$ \\
$V_{ox}$ & úbytok napätia na oxidovej vrstve štruktúry MOS & $V$ \\
$w$ & šírka oblasti priestorového náboja & $m$ \\
$x$ & vzdialenosť & $m$ \\
$\overline z$ & stredná hodnota náhodnej premennej z & \\
$\delta z$ & rozptyl náhodnej premennej z & \\
$z^{'}$ & priestorová derivácia náhodnej premennej z & \\

& & \\ % Gap to separate the Roman symbols from the Greek
$\epsilon$ & permitivita & $Fm^{-1}$ \\
$\epsilon_s$ & permitivita $Si$ & $Fm^{-1}$ \\
$\epsilon_{ox}$ & permitivita $SiO_2$  & $Fm^{-1}$ \\
$\varphi$ & elektrický potenciál & $V$ \\
$\varphi_{ms}$ & rozdiel výstupných potenciálov kovu a polovodiča & $V$ \\
$\varphi_{s}$ & úbytok napätia na vrstve polovodiča (povrchový potenciál) & $V$ \\
$\mu_{n}$ & pohyblivosť elektrónov v polovodiči & $m^2V^{-1}s^{-1}$ \\
$\mu_{p}$ & pohyblivosť dier v polovodiči & $m^2V^{-1}s^{-1}$ \\
$\omega$ & uhlová frekvencia & $s^{-1}$ \\
$\tau_g$ & generačná doba minoritných nosičov náboja & $s$ \\

% Symbol & Name & Unit \\
}
