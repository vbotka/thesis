% Appendix A

\chapter{Numerické riešenie Poissonovej rovnice.} % Main appendix title

\label{app:AppendixA} % For referencing this appendix elsewhere, use \ref{AppendixA}

\lhead{Appendix A. \emph{Numerické riešenie Poissonovej rovnice}} % This is for the header on each page - perhaps a shortened title

Poissonovu rovnicu možno napísať v normalizovanom tvare (Dodatok
\ref{app:AppendixB})

\begin{equation}\label{eq:A.1}
{\frac{d^2u}{dx^2} = e^u - e^{2u_f-u} + \alpha(x) - 1} \qquad {x\ge0}
\end{equation}

Koncentrácie majoritných a minoritných nosičov náboja predstavujú prvé
dva členy na pravej strane rovnice. Koncentráciu substrátu a prímesí
predstavuje člen $\alpha(x)-1$. Uvedenú rovnicu budeme riešiť ako
diferenciálnu rovnicu druhého rádu s počiatočnými podmienkami, ktoré
získame následujúcou úvahou \cite{App.1}. Vo väčšine prípadov
nehomogénnej koncentrácie polovodiča možno nájsť hĺbku v polovodiči,
za ktorou môžeme považovať koncentráciu prímesí za konštantnú. Označme
túto hĺbku $x_1$ a určime ju z podmienky

\begin{equation}\label{eq:A.2}
{\alpha(x_1)  =  0.01}
\end{equation}

Potom pre $x\ge{x_1}$  môžeme zanedbať člen $\alpha(x)$

\begin{equation}\label{eq:A.3}
{\frac{d^2u}{dx^2} = e^u -  e^{2u_f-u}  - 1} \qquad {x\ge{x_1}}
\end{equation}

Vzťah \ref{eq:A.3} predstavuje
Poissonovu rovnicu pre homogénny polovodič, ktorú možno riešiť
analyticky s okrajovými podmienkami

\begin{equation}\label{eq:A.4}
u(\infty) = 0 \qquad \frac{du}{dx}\Big\rvert_{x=\infty} = 0
\end{equation}

aby sme dostali vzťah pre prvú deriváciu potenciálu

\begin{equation}\label{eq:A.5}
\frac{du}{dx} = - \frac{u}{|u|} \sqrt{2} \Big[e^u - e^{2u_f-u} - 1
  \Big]^{\frac{1}{2}} \qquad {x\ge{x_1}}
\end{equation}

Počiatočné podmienky pre riešenie rovnice \ref{eq:A.1} v oblasti
${0\leq{x}\leq{x_1}}$ potom tvorí voľne zvolený potenciál v bode $x_1$
a prvá derivácia potenciálu v bode $x_1$ vyjadrená vzťahom
\ref{eq:A.5}. Vhodnou volbou potenciálu v bode $x_1$ a opakovaným
numerickým riešením rovnice \ref{eq:A.1} z objemu po povrch polovodiča
dostaneme súbor priebehov potenciálu v polovodiči od akumulácie po
inverziu. Potrebujeme ešte poznať napätie hradla pre každý priebeh
potenciálu v polovodiči pri známej kapacite oxidovej vrstvy.  Pre
normálové zložky intenzity elektrického poľa na rozhraní oxidu a
polovodiča platí vzťah

\begin{equation}\label{eq:A.6}
\varepsilon_{ox}E_{ox} = \varepsilon_s{E_s}
\end{equation}

Ak označíme hrúbku oxidovej vrstvy $h_{ox}$, pre napätie hradla
dostaneme vzťahy

\begin{equation}\label{eq:A.7}
v_g = u_s + h_{ox}E_{ox}
\end{equation}

\begin{equation}\label{eq:A.8}
v_g = u_s - ku_{s}^{'} \qquad k =
h_{ox}\frac{\varepsilon_s}{\varepsilon_{ox}}
\end{equation}

kde $u_s$ je hodnota potenciálu na povrchu polovodiča a $u_{s}^{'}$
jej priestorová derivácia. Po odnormovaní sme tým získali priebeh
povrchového potenciálu ako funkciu napätia hradla (obrázok \ref{fig:1.2}). Pre
výpočet kapacity štruktúry MOS následovným vzťahom (Dodatok
\ref{app:AppendixC})

\begin{equation}\label{eq:A.9}
\frac{C_{mos}}{C_{ox}} = 1 - \frac{du_s}{dv_g}
\end{equation}

potrebujeme poznať hodnotu derivácie povrchového potenciálu podľa
napätia hradla. Derivovaním rovnice \ref{eq:A.1} podľa $v_g$ dostaneme
vzťah

\begin{subequations}\label{eq:A.10}
\begin{align}
\frac{d^{2}w}{dx^2} &= w \Big[e^u + e^{2u_f-u}\Big] \qquad &{x\ge{0}} \label{subeq:A.10a}\\[0.5cm]
\intertext{kde $$w = \frac{du}{dv_g}$$}
\intertext{a tým istým postupom pre \ref{eq:A.5} a \ref{eq:A.8} dostaneme vzťahy}
\frac{dw}{dx}\frac{du}{dx} &= w \Big[e^u - e^{2u_f-u} - 1\Big] \qquad &{x\ge{x_1}} \label{subeq:A.10b}\\[0.5cm]
w_s - kw_s^{'} &= 1 \label{subeq:A.10c}
\end{align}
\end{subequations}

Pre volne zvolenú hodnotu $w$ v bode $x_1$ vypočítame prvú deriváciu
podľa vzťahu \ref{subeq:A.10b} a s oboma počiatočnými podmienkami
riešime rovnicu \ref{subeq:A.10a} z bodu $x_1$ smerom k povrchu, čím
získame hodnoty $\beta$ a $\gamma$ pre veličiny $w_s$ a $w^{'}_s$.
Pretože hodnota w v bode $x_1$ bola voľne zvolená, nemusia hodnoty
$\beta$ a $\gamma$ spĺňať podmienku \ref{subeq:A.10c}. Pretože rovnice
\ref{subeq:A.10a}, \ref{subeq:A.10b} a \ref{subeq:A.10c} sú lineárne,
platia vzťahy

\begin{subequations}\label{eq:A.11}
\begin{equation}
1 = w_s - kw^{'}_s = \frac{\beta - k\gamma}{\beta - k\gamma} = \frac{\beta}{\beta -k\gamma} - k\frac{\gamma}{\beta -k\gamma} \label{subeq:A.11a}\\[0.5cm]
\end{equation}
\begin{equation}
w_s = \frac{\beta}{\beta -k\gamma} \qquad w^{'}_s = \frac{\gamma}{\beta -k\gamma} \label{subeq:A.11b}
\end{equation}
\end{subequations}

Ako výsledok možno vypočítať kapacitu štruktúry MOS

\begin{equation}\label{eq:A.12}
\frac{C_{mos}}{C_{ox}} = 1 - \frac{du_s}{dv_g} = 1 - w_s = -kw^{'}_s = -k\frac{\gamma}{\beta - k\gamma}
\end{equation}

Treba poznamenať, že kapacita vypočítaná podľa vzťahov \ref{eq:A.10} a
\ref{eq:A.12} je nízkofrekvenčná kapacita štruktúry MOS, pretože vo
vzťahoch \ref{eq:A.10} je započítaný príspevok minoritných nosičov
náboja. Vysokofrekvenčnú kapacitnú závislosť dostaneme elimináciou
členov predstavujúcich príspevok minoritných nosičov náboja zo vzťahov
\ref{eq:A.10}. Dostaneme

\begin{subequations}\label{eq:A.13}
\begin{align}
\frac{d^{2}w}{dx^2} &= we^u \qquad &{x\ge{x_1}}\label{subeq:A.13a}\\[0.5cm]
\frac{dw}{dx}\frac{du}{dx} &= w \Big[e^u - 1\Big] \qquad &{x\ge{x_1}}\label{subeq:A.13b}\\[0.5cm]
w_s - kw_s^{'} &= 1\label{subeq:A.13c}
\end{align}
\end{subequations}

Pre výpočet kapacitnej závislosti štruktúry MOS v stave  hlbokého ochudobnenia treba eliminovať  príspevky  minoritných nosičov náboja aj zo vzťahov pre výpočet potenciálu. Úpravou \ref{eq:A.1} a \ref{eq:A.5} dostaneme

\begin{subequations}\label{eq:A.14}
\begin{align}
\frac{d^2u}{dx^2} &= {e^u + \alpha(x) - 1} \qquad &{x\ge0}\label{subeq:A.14a}\\[0.5cm]
\frac{du}{dx} &= - \frac{u}{|u|} \sqrt{2} \Big[e^u - 1\Big]^{\frac{1}{2}} \qquad &{x\ge{x_1}}\label{subeq:A.14b}
\end{align}
\end{subequations}

Na obrázku \ref{fig:1.3} sú znázornené nízkofrekvenčná,
vysokofrekvenčná CV krivka a CV krivka hlbokého ochudobnenia,
vypočítané uvedeným postupom. Pre riešenie diferenciálnej rovnice bola
použitá metóda prediktor-korektor so štartovacím úsekom
Runge-Kutta. Program bol napísaný v jazyku Fortran a výpočet jednej CV
závislosti s približne 100 bodmi trval na počítacoch ADT 4500
resp. IBM PC AT s matematickým koprocesorom 1 až 2 minúty v
dvojnásobnej presnosti operácií s plávajucou čiarkou.
