% Appendix A

\chapter{Numerical solution of the Poisson equation.}\label{app:AppendixA}
\raggedright\lhead{Appendix A. \emph{Numerical solution of Poisson's equation}}

The Poisson equation can be written in the normalized form
(Appendix~\ref{app:AppendixB})

\begin{equation}\label{eq:A.1}
  \frac{d^2u}{dx^2} = e^u - e^{2u_f-u} + \alpha(x) - 1 \qquad {x\ge0}
\end{equation}

The concentrations of the majority and minority charge carriers are
the first two terms on the right-hand side of the equation. The
substrate concentration and impurities is represented by the term
$\alpha(x)-1$. We will solve the above equation as a second order
differential equation with initial conditions, which we obtain by the
following reasoning~\cite{App.1}. In most cases of inhomogeneous
semiconductor concentration, the depth in the semiconductor can be
found, beyond which we can consider the concentration of impurities to
be constant. Let us denote this depth by $x_1$ and determine it from
the condition

\begin{equation}\label{eq:A.2}
  \alpha(x_1) = 0.01
\end{equation}

Then for $x\ge{x_1}$ we can neglect the $\alpha(x)$ term

\begin{equation}\label{eq:A.3}
  \frac{d^2u}{dx^2} = e^u - e^{2u_f-u} - 1 \qquad {x\ge{x_1}}
\end{equation}

The relation~\ref{eq:A.3} represents the Poisson equation for a
homogeneous semiconductor, which can be solved analytically with
boundary conditions

\begin{equation}\label{eq:A.4}
  u(\infty) = 0 \qquad \frac{du}{dx}\Big\rvert_{x=\infty} = 0
\end{equation}

to get the relation for the first derivative of the potential

\begin{equation}\label{eq:A.5}
  \frac{du}{dx} = - \frac{u}{|u|} \sqrt{2} \big{[e^u - e^{2u_f-u} - 1 \big]}^{\frac{1}{2}} \qquad {x\ge{x_1}}
\end{equation}

Initial conditions for solving the equation~\ref{eq:A.1} in the domain
${0\leq{x}\leq{x_1}}$ then forms a freely chosen potential at the
point $x_1$ and the first derivative of the potential at the point
$x_1$ expressed by the relation~\ref{eq:A.5}. By an appropriate choice
of the potential at $x_1$ and by repeated numerical solution of the
equation~\ref{eq:A.1} from the volume after the surface of the
semiconductor, we obtain a set of potential waveforms in the
semiconductor from accumulation to inversion. We still need to know
the gate voltage for each potential waveform in the semiconductor at a
known oxide capacitance layer.  For the normal components of the
electric field strength at the interface of the oxide and the
semiconductor, the relation

\begin{equation}\label{eq:A.6}
  \varepsilon_{ox}E_{ox} = \varepsilon_s{E_s}
\end{equation}

If we denote the thickness of the oxide layer by $h_{ox}$, for the
gate voltage we get the relations

\begin{align}
  v_g &= u_s + h_{ox}E_{ox} \label{eq:A.7} \\
  v_g &= u_s - ku_{s}^{'} \qquad\qquad where {k = h_{ox}\frac{\varepsilon_s}{\varepsilon_{ox}}} \label{eq:A.8}
\end{align}

, where $u_s$ is the value of the potential at the surface of the
semiconductor and $u_{s}^{'}$ its spatial derivative. After
unnormalizing, we have thus obtained the waveform of the surface
potential as a function of the gate voltage Figure~\ref{fig:1.2}. To
calculate the capacitance of the MOS structure by the following
relation (Appendix~\ref{app:AppendixC})

\begin{equation}\label{eq:A.9}
  \frac{C_{mos}}{C_{ox}} = 1 - \frac{du_s}{dv_g}
\end{equation}

we need to know the value of the derivative of the surface potential
by the gate voltage. Derivation of the equation~\ref{eq:A.1} by $v_g$
gives relation

\begin{samepage}
  \begin{subequations}\label{eq:A.10}
    \begin{align}
      \frac{d^{2}w}{dx^2} &= w \Big[e^u + e^{2u_f-u}\Big] \qquad &{x\ge{0}} \label{eq:A.10a} \\
      \intertext{,where $w=\frac{du}{dv_g}$}
      \intertext{and by the same procedure for~\ref{eq:A.5} and~\ref{eq:A.8} we get the relations}
      \frac{dw}{dx}\frac{du}{dx} &= w \Big[e^u - e^{2u_f-u} - 1\Big] \qquad &{x\ge{x_1}} \label{eq:A.10b} \\[0.3cm]
      w_s - kw_s^{'} &= 1 \label{eq:A.10c}
    \end{align}
  \end{subequations}
\end{samepage}

For a freely chosen value of $w$ at $x_1$, we calculate the first
derivative according to the equation~\ref{eq:A.10b} and with both
initial conditions we solve equation~\ref{eq:A.10a} from the point
$x_1$ towards the surface, thus obtaining values of $\beta$ and
$\gamma$ for the quantities $w_s$ and $w^{'}_s$.  Because the value of
w at point $x_1$ has been freely chosen, the values of $\beta$ and
$\gamma$ satisfy the condition~\ref{eq:A.10c}. Because
equations~\ref{eq:A.10a},~\ref{eq:A.10b} and~\ref{eq:A.10c} are
linear, the relations

\begin{samepage}
  \begin{subequations}\label{eq:A.11}
    \begin{align}
      1 &= w_s - kw^{'}_s = \frac{\beta - k\gamma}{\beta - k\gamma} = \frac{\beta}{\beta -k\gamma} - k\frac{\gamma}{\beta -k\gamma} \\[0.3cm]
      w_s &= \frac{\beta}{\beta -k\gamma} \qquad w^{'}_s = \frac{\gamma}{\beta -k\gamma}
    \end{align}
  \end{subequations}
\end{samepage}

As a result, the capacitance of the MOS structure can be calculated

\begin{equation}\label{eq:A.12}
  \frac{C_{mos}}{C_{ox}} = 1 - \frac{du_s}{dv_g} = 1 - w_s = -kw^{'}_s = -k\frac{\gamma}{\beta - k\gamma}
\end{equation}

It should be noted that the capacitance calculated by the relations~\ref{eq:A.10}
and~\ref{eq:A.12} is the low-frequency capacitance of the MOS structure, because in
relations~\ref{eq:A.10} the contribution of the minority carriers is accounted for
charge. The high-frequency capacitance dependence is obtained by eliminating
the terms representing the contribution of the minority charge carriers from
relations~\ref{eq:A.10}. We obtain

\begin{samepage}
  \begin{subequations}\label{eq:A.13}
    \begin{align}
      \frac{d^{2}w}{dx^{2}} &= we^{u} \qquad &{x\ge{x_1}}                 \label{eq:A.13a} \\[0.3cm]
      \frac{dw}{dx}\frac{du}{dx} &= w \big[e^u-1\big] \qquad &{x\ge{x_1}} \label{eq:A.13b} \\[0.3cm]
      w_s-kw_s^{'} &= 1 \label{eq:A.13c}
    \end{align}
  \end{subequations}
\end{samepage}

To calculate the capacitance dependence of the MOS structure in the
deep state depletion, we need to eliminate the contributions of both
minority charge carriers and from the relations for potential
calculation. By modifying~\ref{eq:A.1} and~\ref{eq:A.5} we obtain

\begin{samepage}
  \begin{subequations}\label{eq:A.14}
    \begin{align}
    \frac{d^2u}{dx^2} &= {e^u + \alpha(x)-1} \qquad &{x\ge0}                                     \label{eq:A.14a} \\[0.3cm]
    \frac{du}{dx} &= -\frac{u}{|u|} \sqrt{2} \big{[e^u-1\big]}^{\frac{1}{2}} \qquad &{x\ge{x_1}} \label{eq:A.14b}
    \end{align}
  \end{subequations}
\end{samepage}

The figure~\ref{fig:1.3} shows the low-frequency, high-frequency CV
curve, and the deep depletion CV curve, calculated by the above
procedure. To solve the differential equation, the the
predictor-corrector method was used with a starting section
Runge-Kutta. The program was written in Fortran and the calculation of
one CV dependence with approximately 100 points took on an ADT 4500 or
IBM PC AT with mathematical coprocessor 1 to 2 minutes in double the
precision of floating point operations.
