% Appendix H

\chapter{Programs for data acquisition, processing and display for area distribution of MOS structure parameters.}\label{app:AppendixH}
\lhead{Appendix H. \emph{Data acquisition, processing and display programs}}

\begin{verbatim}
I. DATA COLLECTION PROGRAMS.

  All data acquisition programs use a stepper to control the programs
  to control the device:

   * ZONDUP.EXE - table stroke
   * ZONDDN.EXE - table start
   * ZONDST.EXE - table feed

   They read information about the movement of the stepper from the Z.XY subfile

1. ZCT.EXE      - Main program of the CCT method
   Segments:
   * ZCT1.EXE   - initialization of GPIB bus and devices
   * ZCT2.EXE   - measurement of V(t) and saving dV/dt to the subfile
   * ZCT9.EXE   - putting the GPIB bus into the flood state

2. ZHF.EXE      - Main program of equilibrium and non-equilibrium HF CV method
   Segments:
   * ZHF1.EXE   - initialization of GPIB bus and devices
   * ZHF2.EXE   - C(V ) measurement
   * ZHF3.EXE   - smoothing of measured data and saving to a subfile
   * ZHF9.EXE   - setting the GPIB bus to the flood state

3. ZLF.EXE      - Main program of quasi-static CV method
   Segments:
   * ZLF1.EXE   - initialization of GPIB bus and devices
   * ZLF2.EXE   - measurement of the C(V ) dependence
   * ZLF3.EXE   - smoothing of the measured data, interpolation and saving to a subfile
   * ZLF9.EXE   - putting the GPIB bus into a flood state

4. ZOXHF.EXE    - Measurements of the oxide capacity using the HF CV method
   Segments:
   * ZOXHF1.EXE - initialization of GPIB bus and instruments
   * ZOXHF2.EXE - oxide layer capacitance measurement and saving to subfile
   * ZOXHF9.EXE - putting the GPIB bus into the trigger state

5. ZOXLF.EXE    - Oxide capacitance measurements using quasi-static CV method
   Segments:
   * ZLF1.EXE   - Initialization of GPIB bus and instruments
   * ZOXLF2.EXE - oxide layer capacitance measurements and storage in a subfile
   * ZLF9.EXE   - resetting the GPIB bus to its original state


II. DATA PROCESSING PROGRAMS.

6.    ZNX.EXE - calculation of concentration profile N(x)
7.    ZNB.EXE - calculation of substrate concentration N
8.    ZFV.EXE - calculation of surface potential as a function of 
                of the gate voltage f (V )
9.    ZDF.EXE - calculation of interface trap density D
10.   ZTX.EXE - calculation of generation time of minority lifetime 
                of charge carriers t (x)
11.  ZWOX.EXE - calculation of the oxide roughness h
12.  ZYXI.EXE - calculation of integral value Y(x)
13.  ZYXM.EXE - calculation of the mean value, standard deviation
                error and linear regression
14.ZKORFF.EXE - calculation of the correlation coefficient between para-
                meters from Subor1 <> Subor2
15.ZKORYX.EXE - calculation of correlation coefficient of dependence Y(x)


III. GRAPHICAL DATA DISPLAY PROGRAMS.

16.   ZGF.EXE - display of the dependence Y(x) with a description of the axis
17. ZSURF.EXE - full screen display of the distribution of parameters and
                functional dependencies Y(x), for different x
18. ZVIEW.EXE - display of individual Y(x) dependencies
19.ZVIEWD.EXE - display of individual Y1(x) and Y2(x) dependencies


IV. AUXILIARY PROGRAMS.

20.   BELL.EXE - alarm
21. ZASCII.EXE - conversion of binary format to ASCII
22. ZERR.EXE   - printout of positions on the board with erroneous measured data
23. ZREPLC.EXE - exchange of records in the data subfile
24. ZTRUNC.EXE - short-circuit of data subfile
\end{verbatim}
