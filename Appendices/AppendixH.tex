% Appendix H

\chapter{Programy zberu, spracovania a zobrazenia dát pre plošné rozloženie parametrov štruktúr MOS.} % Main appendix title

\label{app:AppendixH} % For referencing this appendix elsewhere, use \ref{AppendixA}

\lhead{Appendix H. \emph{Programy zberu, spracovania a zobrazenia dát}} % This is for the header on each page - perhaps a shortened title

\begin{verbatim}
I. PROGRAMY ZBERU DAT.

Vsetky programy zberu dat pouzivaju pre riadenie krokovacieho zariadenia programy:
* ZONDUP.EXE - zdvih stolika
* ZONDDN.EXE - spustenie stolika
* ZONDST.EXE - posuv stolika
Informacie o pohybe krokovacieho zariadenia citaju zo suboru Z.XY

1. ZCT.EXE - hlavny program metody CCT
   segmenty:
   * ZCT1.EXE - inicializacia zbernice IMS-2 a pristrojov
   * ZCT2.EXE - meranie zavislosti V (t) a ukladanie dV /dt do
     vystupneho suboru
   * ZCT9.EXE - uvedenie zbernice IMS-2 do ppvodneho stavu

2. ZHF.EXE - hlavny program rovnovaznej a nerovnovaznej HF CV metody
   segmenty:
   * ZHF1.EXE - inicializacia zbernice IMS-2 a pristrojov
   * ZHF2.EXE - meranie zavislosti C   (V )
   * ZHF3.EXE - vyhladenie nameranych dat a ulozenie do vystupneho
     suboru
   * ZHF9.EXE - uvedenie zbernice IMS-2 do ppvodneho stavu

3. ZLF.EXE - hlavny program kvazistatickej CV metody
   segmenty:
   * ZLF1.EXE - inicializacia zbernice IMS-2 a pristrojov
   * ZLF2.EXE - meranie zavislosti C   (V )
   * ZLF3.EXE - vyhladenie nameranych dat, interpolacia a ulozenie do
     vystupneho suboru
   * ZLF9.EXE - uvedenie zbernice IMS-2 do ppvodneho stavu

4. ZOXHF.EXE - hlavny program merania kapacity oxidu pomocou HF
   CV metody
   segmenty:
   * ZOXHF1.EXE - inicializacia zbernice IMS-2 a pristrojov
   * ZOXHF2.EXE - meranie kapacity oxidovej vrstvy a ukladanie do
     vystupneho suboru
   * ZOXHF9.EXE - uvedenie zbernice IMS-2 do ppvodneho stavu

5. ZOXLF.EXE - hlavny program merania kapacity oxidu pomocou
   kvazistatickej CV metody
   segmenty:
   * ZLF1.EXE - inicializacia zbernice IMS-2 a pristrojov
   * ZOXLF2.EXE - meranie kapacity oxidovej vrstvy a ukladanie do
     vystupneho suboru
   * ZLF9.EXE - uvedenie zbernice IMS-2 do ppvodneho stavu


II. PROGRAMY SPRACOVANIA DAT.

6.    ZNX.EXE - vypocet koncentracneho profilu N(x)
7.    ZNB.EXE - vypocet koncentracie substratu N
8.    ZFV.EXE - vypocet povrchoveho potencialu ako funkcie 
                napbtia hradla f (V )
9.    ZDF.EXE - vypocet hustoty pasci rozhrania D
10.   ZTX.EXE - vypocet generacneho casu zivota minoritnych 
                nosicov naboja t (x)
11.  ZWOX.EXE - vypocet hrubky oxidu h
12.  ZYXI.EXE - vypocet integralnej hodnoty Y(x)
13.  ZYXM.EXE - vypocet strednej hodnoty, smerodajnej od-
                chylky a linearnej regresie
14.ZKORFF.EXE - vypocet korelacneho koeficientu medzi para-
                metrami zo suborov Subor1 <> Subor2
15.ZKORYX.EXE - vypocet korelacneho koeficientu zavislosti
                Y(x)


III. PROGRAMY GRAFICKEHO ZOBRAZENIA DAT.

16.   ZGF.EXE - zobrazenie zavislosti Y(x) s popisom osi
17. ZSURF.EXE - plosne zobrazenie rozlozenia parametrov a
                funkcnych zavislosti Y(x), pre rpzne x
18. ZVIEW.EXE - zobrazenie jednotlivych zavislosti Y(x)
19.ZVIEWD.EXE - zobrazenie jednotlivych zavislosti Y1(x) a
                Y2(x)


IV. POMOCNE PROGRAMY.

20.   BELL.EXE - alarm
21. ZASCII.EXE - konverzia binarneho formatu na ASCII
22.   ZERR.EXE - vypis pozicii na doske s chybne nameranymi
                 datami
23. ZREPLC.EXE - vymena zaznamov v datovom subore
24. ZTRUNC.EXE - skratenie datoveho suboru
\end{verbatim}
