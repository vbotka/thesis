% Appendix F

\chapter{Calculation of the surface potential $\varphi_s$ from Q-C method.}\label{app:AppendixF}
\lhead{Appendix F. \emph{Calculation of surface potential $\varphi_s$ from Q-C method}}

The method for determining the surface potential of an MOS structure
is described in Appendix~\ref{app:AppendixC}. Here we give its main idea.

To change the charges on a series-parallel circuit of Q-C capacitors
methods we can write (see Appendix~\ref{app:AppendixE}).

\begin{equation}\label{eq:F.1}
  \Delta Q_x + \Delta Q_i = \Delta Q_w + \Delta Q_{mos}
\end{equation}

If we express the change of charge on voltage-independent capacitors
by their capacitance and voltage, assuming that we start from the
state where $V_i=0$ and $V_g=0$, we can write

\begin{equation}\label{eq:F.2}
  \Delta Q_{mos} = (C_{iLF} + C_{x})V_{i} - C_{w}V_{g}
\end{equation}

We can also express the change of charge on the MOS structure by its
own parameters

\begin{equation}\label{eq:F.3}
  \Delta Q_{mos} = C_{ox}(V_{g} - \varphi_{s} + \varphi_{s0})
\end{equation}

and by combining~\ref{eq:F.2} and~\ref{eq:F.3} we can write the
resulting relation

\begin{equation}\label{eq:F.4}
  \varphi_{s} = \varphi_{s0} - \frac{C_{iLF} + C_{x}}{C_{ox}}V_i + {\bigg[1 + \frac{C_{w}}{C_{ox}}\bigg]}V_{g}
\end{equation}
