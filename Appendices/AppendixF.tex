% Appendix F

\chapter{Určenie povrchového potenciálu $\varphi_s$ z Q-C metódy.} % Main appendix title

\label{app:AppendixF} % For referencing this appendix elsewhere, use \ref{AppendixA}

\lhead{Appendix F. \emph{Určenie povrchového potenciálu $\varphi_s$ z Q-C metódy}} % This is for the header on each page - perhaps a shortened title

Spôsob určenia povrchového potenciálu štruktúry MOS je popísany v článku \cite{App.3}. Tu uvedieme jeho hlavnú myšlienku.

Pre zmenu nábojov na sériovo-paralelnom zapojení kondenzátorov Q-C
metódy môžeme písať (pozri dodatok \ref{app:AppendixE}).

\begin{equation}\label{eq:F.1}
\Delta Q_x + \Delta Q_i = \Delta Q_w + \Delta Q_{mos}
\end{equation}

Ak vyjadríme zmenu náboja na napäťovo-nezávislých kondenzátoroch
pomocou ich kapacity a napätia, za predpokladu, že vychádzame zo stavu
kedy $V_i=0$ a $V_g=0$, môžeme písať

\begin{equation}\label{eq:F.2}
\Delta Q_{mos} = (C_{iLF} + C_{x})V_{i} - C_{w}V_{g}
\end{equation}

Zmenu náboja na štruktúre MOS môžeme vyjadriť aj pomocou jej vlastných
parametrov

\begin{equation}\label{eq:F.3}
\Delta Q_{mos} = C_{ox}(V_{g} - \varphi_{s} + \varphi_{s0})
\end{equation}

a kombináciou \ref{eq:F.2} a \ref{eq:F.3} môžeme písať výsledný vzťah

\begin{equation}\label{eq:F.4}
\varphi_{s} = \varphi_{s0} - \frac{C_{iLF} + C_{x}}{C_{ox}}V_i + \Big[1 + \frac{C_{w}}{C_{ox}}\Big]V_{g}
\end{equation}
