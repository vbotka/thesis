% Appendix D

\chapter{Termodynamická rovnováha v nehomogénne dotovanom substráte.}\label{app:AppendixD}
\lhead{Dodatok D. \emph{Termodynamická rovnováha v nehomogénne dotovanom substráte}}

V prípade termodynamickej rovnováhy platí pre elektrónovú zložku prúdu
vzťah

\begin{equation}\label{eq:D.1}
  I_n = qD_n\frac{dn(x)}{dx} - q\mu_{n}n(x) \frac{d\varphi(x)}{dx} = 0
\end{equation}

Z tohoto vzťahu, použitím Einsteinovho vzťahu, možno vyjadriť
intenzitu elektrického poľa

\begin{equation}\label{eq:D.2}
  E(x) = - \frac{kT}{q} \frac{1}{n(x)} \frac{dn(x)}{dx}
\end{equation}

Keďže priestorový náboj je v tomto prípade určený ionizovanými donormi
$N_D$ a majoritnými elektrónmi, Poissonova rovnica nadobúda tvar

\begin{equation}\label{eq:D.3}
  \frac{dE(x)}{dx} = \frac{q}{\epsilon} \big[N_D(x) - n(x)\big]
\end{equation}

Deriváciou rovnice~\ref{eq:D.2} a porovnaním s rovnicou~\ref{eq:D.3}
dostaneme výraz pre výpočet koncentračného profilu dotujúcich atómov z
profilu majoritných nosičov náboja

\begin{equation}\label{eq:D.4}
  N_D(x) = n(x) - \frac{kT\epsilon}{q^2} \frac{d}{dx} \bigg[\frac{1}{n(x)} \frac{dn(x)}{dx}\bigg]
\end{equation}
