% Appendix B

\chapter{Úprava Poissonovej rovnice do normalizovaného tvaru.}\label{app:AppendixB}
\lhead{Dodatok B. \emph{Úprava Poissonovej rovnice do normalizovaného tvaru}}

Poissonova rovnica má tvar

\begin{equation}\label{eq:B.1}
  \frac{d^{2}\varphi}{dx^2} = - \frac{\rho(x)}{\epsilon}
\end{equation}

Ak uvažujeme, že v polovodiči s koncentráciou substrátu $N_b$
(predpokladajme donory) je vytvorený nehomogénny koncentračný profil
$N(x)$ (predpokladajme akceptory), môžeme pre hustotu náboja napísať
vzťah

\begin{equation}\label{eq:B.2}
  \rho(x) = - q (n(x) - p(x) + N(x) -N_b)
\end{equation}

Členy $n(x)$ a $p(x)$, ktoré predstavujú voľné nosiče náboja,
vyjadríme pomocou normalizovaných potenciálov $u$ a $u_f$

\begin{subequations}\label{eq:B.3}
  \begin{align}
    u(x) &= \frac{E_i(\infty) - E_i(x)}{kT} = \frac{q\varphi(x)}{kT} \label{eq:B.3a}\\[0.3cm]
    u_f &= \frac{E_i(\infty) - E_f}{kT} \label{eq:B.3b}
  \end{align}
\end{subequations}

a pomocou koncentrácie substrátu

\begin{subequations}\label{eq:B.4}
  \begin{align}
    n(x) &= N_b {e}^{u(x)}\label{eq:B.4a}\\[0.3cm]
    p(x) &= n_i {e}\frac{E_i(x) - E_f}{kT} = n_i {e}^{u_f-u(x)} = N_b {e}^{2u_f-u(x)}\label{eq:B.4b}\\[0.3cm]
    \intertext{keď sme použili vzťah pre koncentráciu elektrónov v substráte}
    n(\infty) &= N_b = n_i {e}\frac{E_f-E_i(\infty)}{kT} = n_i {e}^{-u_f}\label{eq:B.4c}
\end{align}
\end{subequations}

Potom hustotu náboja môžeme napísať v tvare

\begin{equation}\label{eq:B.5}
  \rho(x) = -qN_b(e^{u(x)} - e^{2u_f-u(x)} + \alpha(x) - 1)
\end{equation}

kde $\alpha(x) = \frac{N(x)}{N_b}$

Dosadením hustoty náboja~\ref{eq:B.5} a substitúcie~\ref{eq:B.3} do rovnice~\ref{eq:B.1} dostaneme

\begin{equation}\label{eq:B.6}
  \frac{d^{2}u(x)}{dx^2} = \frac{q^2N_b}{kT\varepsilon}(e^{u(x)} - e^{2u_f-u(x)} + \alpha(x) - 1)
\end{equation}

Zavedieme efektívnu Debayovu dĺžku

\begin{equation}\label{eq:B.7}
  L_D = {\Bigg[\frac{kT\varepsilon}{q^{2}N_b}\Bigg]}^{\frac{1}{2}}
\end{equation}

na ktorú vzdialenosť $x$ normujeme

\begin{equation}\label{eq:B.8}
  \xi = \frac{x}{L_D}
\end{equation}

Môžeme písať Poissonovu rovnicu v normovanom tvare

\begin{equation}\label{eq:B.9}
  \frac{d^{2}u(\xi)}{d\xi^2} = e^{u(\xi)} - e^{2u_f-u(\xi)} + \alpha(\xi) - 1
\end{equation}
