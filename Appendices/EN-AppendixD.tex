% Appendix D

\chapter{Thermodynamic equilibrium in inhomogeneously endowed substrate.}\label{app:AppendixD}
\lhead{Appendix D. \emph{Thermodynamic equilibrium in an inhomogeneously endowed substrate}}

In the case of thermodynamic equilibrium, the electron component of
the current relation

\begin{equation}\label{eq:D.1}
  I_n = qD_n\frac{dn(x)}{dx} - q\mu_{n}n(x) \frac{d\varphi(x)}{dx} = 0
\end{equation}

From this relation, using Einstein's relation, we can express the
electric field strength

\begin{equation}\label{eq:D.2}
  E(x) = - \frac{kT}{q} \frac{1}{n(x)} \frac{dn(x)}{dx}
\end{equation}

Since the space charge in this case is determined by the ionized
donors $N_D$ and the majority electrons, the Poisson equation takes
the form

\begin{equation}\label{eq:D.3}
  \frac{dE(x)}{dx} = \frac{q}{\epsilon} \big[N_D(x) - n(x)\big]
\end{equation}

By deriving equation~\ref{eq:D.2} and comparing with
equation~\ref{eq:D.3} we obtain an expression for the calculation of
the concentration profile of the interfering atoms from the profile of
the major charge carriers

\begin{equation}\label{eq:D.4}
  N_D(x) = n(x) - \frac{kT\epsilon}{q^2} \frac{d}{dx} \bigg[\frac{1}{n(x)} \frac{dn(x)}{dx}\bigg]
\end{equation}
