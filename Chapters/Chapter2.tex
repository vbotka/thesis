% Chapter 2

\chapter{Objectives of the dissertation.}\label{Chapter2}
\lhead{Chapter 2. \emph{Dissertation Objectives}}
%- - - - - - - - - - - - - - - - - - - - - - - - - - - - - - - - - - -

\begin{enumerate}
\item Building an automated experimental facility for analysis of
  electrophysical properties of MOS structures with inhomogeneous
  distribution of dopants in the substrate, with the ability to
  monitor area distribution of the investigated parameters on the
  silicon substrate.  The workstation includes a high-frequency C-V
  method (including depleted C-V method), low-frequency C-V method,
  low-frequency constant width OPN (CCT) and Q-C method.
\item Automate the methods referred to in point 1.  PC AT computer
  under MS DOS operating system.  To implement methods to use the GPIB
  PCIIA interface, HP4280a measuring instruments, Keithley 642 and a
  Zond A5 tip stepper.
\item The implemented methods are used to determine the depth
  concentration profiles of implanted impurities, oxide layer
  thickness, stress of aligned bands, trap density of the Si-SiO
  interface and depth generation lifetime profile of minority charge
  carriers.  Analyse the problems arising in determining the
  parameters of MOS structures in conjunction with inhomogeneous
  substrate endowment.
\item Determine the depth concentration profiles of active impurities
  and their distributions on the silicon substrate for different
  implantation doses in ranging from $0.6\times{10}^{15}$ to
  $60.0\times{10}^{15}{m}^{-2}$. To investigate how implantation dose
  affects the properties of the Si-SiO interface and the depth
  lifetime profile of minority charge carriers.  Propose a methodology
  for identifying the amount of implanted ions in a semiconductor
  substrate using a capacitive method.
\end{enumerate}
