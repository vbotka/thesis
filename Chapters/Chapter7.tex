% Chapter 7
\chapter{Experimentálne výsledky.}
\label{Chapter7}
\lhead{Chapter 7. \emph{Experimentálne výsledky}}

Záverečný experiment bol uskutočnený na štruktúrach MOS s nehomogénnym
hĺbkovým profilom dotujúcich prímesí, ktorý bol vytvorený procesom
iónovej implantácie s rôznymi dávkami v monokryštále kremíka typu N s
orientáciou [100].

Pred technologickým spracovaním bola otestovaná homogenita
špecifického odporu použitých kremíkových dosiek pomocou zariadenia
Prometrix OmniMap RS35, ktoré využíva štvorbodovú metódu pre určenie
povrchového špecifického odporu. V tabuľke \ref{tab:7.1} sú uvedené
stredné hodnoty špecifického odporu $\overline\rho$ a smerodajnej
odchýlky $\delta\rho$ vyjadrenej absolútnou a relatívnou hodnotou.

\begin{table}[h!]\centering
\begin{tabular}{|c|c|c|c||c|c|c|c|}
\hline
č. & $\overline\rho[\Omega cm]$ & $\delta\rho[\Omega cm]$ & $\delta\rho[\%]$ &
č. & $\overline\rho[\Omega cm]$ & $\delta\rho[\Omega cm]$ & $\delta\rho[\%]$ \\
\hline
1  & 4.3319 & 0.1223 & 2.822 & 11 & 4.5706 & 0.1658 & 3.627 \\
2  & 4.2733 & 0.1204 & 2.817 & 12 & 4.4762 & 0.1860 & 4.155 \\
3  & 5.1040 & 0.3405 & 6.671 & 13 & 4.3332 & 0.1265 & 2.290 \\
4  & 4.6276 & 0.2080 & 4.494 & 14 & 4.8422 & 0.3573 & 7.380 \\
5  & 4.7697 & 0.1824 & 3.824 & 15 & 4.5917 & 0.1741 & 3.791 \\
6  & 4.8007 & 0.2340 & 4.873 & 16 & 4.8134 & 0.2590 & 5.380 \\
7  & 4.2500 & 0.1436 & 3.378 & 17 & 4.4025 & 0.1527 & 3.468 \\
8  & 4.8259 & 0.3163 & 6.554 & 18 & 4.3591 & 0.1290 & 2.960 \\
9  & 4.2853 & 0.1418 & 3.308 & 19 & 4.3877 & 0.1349 & 3.074 \\
10 & 4.2954 & 0.1113 & 2.592 & 20 & 4.5416 & 0.1618 & 3.563 \\
\hline
\end{tabular}
\captionsetup{justification=raggedright, singlelinecheck=false}
{\caption[Stredná hodnota a smerodajná odchýlka špecifického odporu
    testovaných kremíkových dosiek pred technologickým spracovaním]
  {Stredná hodnota a smerodajná odchýlka špecifického odporu
    testovaných kremíkových dosiek pred technologickým
    spracovaním.}\label{tab:7.1}}
\end{table}

Zariadenie Prometrix OmniMap RS35 zmeralo pomocou krokovacieho
zariadenia hodnotu špecifického odporu v 81 bodoch každej dosky. Na
obrázku \ref{fig:7.1} a obrázku \ref{fig:7.2} uvádzame grafické
znázornenie rozloženia špecifického odporu, ktoré je taktiež výstupom
merania uvedeného zariadenia. Body, v ktorých bol zmeraný špecifický
odpor sú na obrázku \ref{fig:7.1} vyznačené znakmi $+$, alebo $-$
podľa toho, či hodnota špecifického odporu v tomto bode ležala nad,
alebo pod strednou hodnotou, ktorá je znázornená hrubšou čiarou.
Predstavu o kvantitatívnom rozložení špecifického odporu si možno
urobiť z trojdimenzionálneho obrázku \ref{fig:7.2}.

Postup hlavných technologických operácií vytvorenia štruktúr MOS na
uvedených substrátoch bol následovný

\begin{itemize}
\item vytvorenie hradloveho oxidu s hrúbkou $100 \nu m$
\item implantácia $P^{31}$ s energiou $120 keV$ a dávkami $0.6, 1.0,
  2.0, 4.0, 5.0, 6.0, 7.0, 8.0, 20.0, 60.0 \times 10^{15} m^{-2}$ pod
  uhlom $7\degree$
\item aktivácia pri teplote $1050 \degree C$ s časovým priebehom: 15
  min. nábeh, 30 min. aktivácia, 40 min. chladenie
\item naparenie Al na obe strany kremíkovej dosky
\item litografický proces vytvorenia CV masky
\item sintrovanie Al FG pri teplote $460 \degree C$ počas 20 min.
\end{itemize}

Uvedeným spôsobom bolo pripravených 20 kremíkových dosiek o priemere 4
palce, vždy dve s rovnakou dávkou implantácie.  V procese zberu dát
bolo na každej kremíkovej doske testovaných 304 štruktúr, pričom
plocha jednej štruktúry bola $0.81 \times 10^{-6} m^{-2}$. Na obrázku
\ref{fig:7.3} sú znázornené priebehy koncentračných profilov
dotujúcich prímesi pre jednotlivé dávky implantácie. Znázornené
priebehy predstavujú strednú hodnotu cez všetky závislosti N(x), ktoré
boli určené na testovanej doske. Z každej dávky je na obrázku
\ref{fig:7.3} zobrazená len jedna kremíková doska.

\begin{figure}[h!]\centering
\includegraphics{Figures/fig-7-1.eps}
\captionsetup{justification=raggedright, singlelinecheck=false}
{\caption[Plošné rozloženie povrchového špecifického odporu kremíkovej
    dosky č.16]{Plošné rozloženie povrchového špecifického odporu
    kremíkovej dosky č.16.}\label{fig:7.1}}
\end{figure}

\begin{figure}[h!]\centering
\includegraphics{Figures/fig-7-2.eps}
\captionsetup{justification=raggedright, singlelinecheck=false}
{\caption[Plošné rozloženie povrchového špecifického odporu kremíkovej
    dosky č.16]{Plošné rozloženie povrchového špecifického odporu
    kremíkovej dosky č.16.}\label{fig:7.2}}
\end{figure}

\begin{figure}[h!]\centering
\includegraphics{Figures/fig-7-3.eps}
\captionsetup{justification=raggedright, singlelinecheck=false}
{\caption[Hĺbkový profil dotujúcich prímesí]{Hĺbkový profil dotujúcich
    prímesí v podpovrchovej oblasti polovodiča vytvorený iónovou
    implantáciou s dávkami $0.6, 1.0, 2.0, 4.0, 5.0, 6.0, 7.0, 8.0,
    20.0, 60.0 \times 10^{15} m^{-2}$. Zobrazené priebehy $N(x)$
    predstavujú strednú hodnotu z priebehov, nameraných na 304
    štruktúrach MOS každej kremíkovej dosky.} \label{fig:7.3}}
\end{figure}
% OBR27.BIT

Tabuľka \ref{tab:7.2} obsahuje číselné hodnoty implantačnej dávky
zadané v procese implantácie $D_{i}$, strednú hodnotu $\overline D$ a
smerodajnú odchýlku $\delta D$ dávok vypočítaných postupom uvedeným v
časti \ref{sec:6.1}.

Pre kontrolu reprodukovateľnosti procesu implantácie boli zmerané
koncentračné profily na ďalších 3 kremíkových doskách.  V tabuľke
\ref{tab:7.3} sa nachádzajú hodnoty dávok implantácie pre tri dvojice
kremíkových dosiek, ktoré boli implantované s tou istou dávkou.

Ako je zrejmé z tabuľky \ref{tab:7.2} a \ref{tab:7.3}, vypočítaná
implantačná dávka je vždy menšia ako dávka zadaná v procese
implantácie.  To je spôsobené jednako tým, že čas implantovaných iónov
je zachytená v oxidovej vrstve a jednako neúplnou aktiváciou
implantovaných iónov v polovodiči. Aby sme určili závislosť medzi
zadanou a vypočítanou dávkou, vypočítali sme lineárnou regresiou
koeficient $b$ vzťahu

\begin{equation}\label{eq:7.1}
\overline D = b D_{i}
\end{equation}

ktorý mal hodnotu $b = 0.71$ a zároveň sme zobrazili závislosť
$\overline D = f(D_{i})$ na obrázku \ref{fig:7.4}.

Tým sme zistili, že z pôvodnej dávky, ktorá bola implantovaná sa stalo
elektricky aktívnymi 71\% implantovaných iónov.

\begin{table}[h!]\centering
\begin{tabular}{|c|c|c|c|}
\hline
č. & $D_{i} 10^{15} [m^{-2}]$ & $\overline D 10^{15} [m^{-2}]$ & $\delta D 10^{15} [m^{-2}]$ \\
\hline
 1 &  0.6 &  0.39 &  0.02 \\
 3 &  1.0 &  0.59 &  0.08 \\
 5 &  2.0 &  1.20 &  0.06 \\
 7 &  4.0 &  2.67 &  0.09 \\
 9 &  5.0 &  3.40 &  0.13 \\
11 &  6.0 &  4.07 &  0.13 \\
13 &  7.0 &  4.72 &  0.14 \\
15 &  8.0 &  5.49 &  0.09 \\
17 & 20.0 & 14.41 &  0.35 \\
19 & 60.0 & 42.63 &  0.21 \\
\hline
\end{tabular}
\captionsetup{justification=raggedright, singlelinecheck=false}
{\caption [Dávky implantácie $D_{i}$] {Dávka implantácie $D_{i}$ ,
    vypočítaná stredná hodnota dávky implantovaných a aktivovaných
    iónov v polovodiči $\overline D$ a jej smerodajná odchýlka $\delta
    D$ na kremíkovej doske.}\label{tab:7.2}}
\end{table}

\begin{table}[h!]\centering
\begin{tabular}{|c|c|c|c|}
\hline
č. & $D_{i} 10^{15} [m^{-2}]$ & $\overline D 10^{15} [m^{-2}]$ & $\delta D 10^{15} [m^{-2}]$ \\
\hline
 9 & 5.0 & 3.40 & 0.13 \\
10 & 5.0 & 3.56 & 0.06 \\
11 & 6.0 & 4.07 & 0.13 \\
12 & 6.0 & 4.03 & 0.12 \\
15 & 8.0 & 5.49 & 0.09 \\
16 & 8.0 & 5.46 & 0.08 \\
\hline
\end{tabular}
\captionsetup{justification=raggedright, singlelinecheck=false}
{\caption[Dávka implantácie $D_{i}$]{Dávka implantácie $D_{i}$,
    vypočítaná stredná hodnota dávky implantovaných a aktivovaných
    iónov v polovodiči $\overline D$ a jej smerodajná odchýlka $\delta
    D$ na kremíkovej doske.}\label{tab:7.3}}
\end{table}

\begin{figure}[h!]\centering
\includegraphics{Figures/fig-7-4.eps}
\captionsetup{justification=raggedright, singlelinecheck=false}
{\caption[Závislosť strednej hodnoty
    $\overline{D}=E(\int(N(x)-N_{b})dx)$ od dávky implantovaných iónov
    $D_{i}$]{Závislosť strednej hodnoty
    $\overline{D}=E(\int(N(x)-N_{b})dx)$ od dávky implantovaných iónov
    $D_{i}$. Zobrazené hodnoty sú rádu $10^{15}$.}\label{fig:7.4}}
\end{figure}
%OBR29.BIT

Aby sme určili stupeň závislosti medzi implantovanou dávkou a
množstvom elektricky aktívnych prímesí v polovodiči, ktoré boli
implantované, vypočítali sme korelačný koeficient medzi týmito
veličinami. Použili sme vzťah

\begin{equation}\label{eq:7.2}
R(X,Y) = \frac{E([X-E(X)][Y-E(Y)])}{D(X)D(Y)}
\end{equation}

ktorý je uvedený napríklad v \cite{7.1}. Vo vzťahu \ref{eq:7.2} X a Y
predstavujú náhodné veličiny, E predstavuje strednú hodnotu a D
označuje smerodajnú odchýlku. Uvedeným spôsobom sme získali hodnotu
korelačného koeficientu

$$R(D_{i}, \overline{D}) = 0.99$$

pričom sme považovali hodnoty $D_{i}$ a $\overline{D}$ za realizácie
náhodnej veličiny a použili sme všetky hodnoty uvedené v tabuľke
\ref{tab:7.2} a {tab:7.3}. Možno poznamenať, že v teórii
pravdepodobnosti je dokázaná veta, podľa ktorej $\rvert R(X,Y)\rvert =
1$ práve vtedy, keď s pravdepodobnosťou 1 platí

$$Y = a + b X$$

Z uvedeného vyplýva, že závislosť medzi hodnotami $D_{i}$ a $\overline
D$ je v tomto prípade lineárna.

Pomocou profesionálneho programu, zakúpeného Teslou Piešťany, na
simuláciu procesu iónovej implantácie boli vypočítané priebehy
koncentrácie prímesi pre dávky 0.6, 5.0 a $60.0 \times 10^{15}
m^{-2}$.  Priebehy koncentračných profilov boli simulované na základe
zadaných podmienok implantácie, pričom sa použila metóda Pearson IV.
Porovnanie nameraných a simulovaných priebehov koncentrácie prímesí je
zobrazené na obrázku \ref{fig:7.5}.

\begin{figure}[h!]\centering
\includegraphics{Figures/fig-7-5.eps}
\captionsetup{justification=raggedright, singlelinecheck=false}
{\caption[Porovnanie stredných hodnôt nameraných priebehov N(x) a
    simulovanych pomocou metódy Pearson IV]{Porovnanie stredných
    hodnôt nameraných priebehov N(x) a simulovanych pomocou metódy
    Pearson IV pre dávky $0.6, 5.0, 60.0 \times 10^{15}
    m^{-2}$.}\label{fig:7.5}}
\end{figure}
% OBR33.BIT

V procese výpočtu hĺbkových profilov dotujúcich prímesí sme zároveň
určili aj hodnoty napätia vyrovnaných pásov $V_{fb}$ pre každú
testovanú štruktúru MOS. Pomocou samostatného programu, ktorý určuje
na základe dát, nachádzajúcich sa v zadanom dátovom súbore strednú
hodnotu a smerodajnú odchýlku uložených parametrov, sme vypočítali
strednú hodnotu $\overline V{fb}$ a smerodajnú odchýlku $\delta
V{fb}$. Zároveň sme pomocou toho istého postupu určili
hodnoty$\overline h_{ox}$ a $\delta h_{ox}$, ktoré sú pre jednotlivé
kremíkové dosky uvedené v tabuľke \ref{tab:7.4}.

\begin{table}[h!]\centering
\begin{tabular}{|c|c|c|c|c|}
\hline
č. & $\overline V_{fb} [V]$ & $\delta V_{fb} [V]$ & $\overline h_{ox} [nm]$ & $\delta h_{ox} [nm]$ \\ 
\hline
 1 & -1.24 & 0.07 &  94.14 & 0.89 \\
 3 & -1.43 & 0.07 &  97.79 & 0.80 \\
 5 & -1.35 & 0.08 &  97.26 & 0.28 \\
 7 & -1.40 & 0.09 &  98.15 & 0.35 \\
 9 & -1.52 & 0.09 & 102.85 & 0.53 \\
11 & -1.48 & 0.08 & 101.65 & 0.32 \\
13 & -1.38 & 0.08 & 100.94 & 0.41 \\
15 & -1.33 & 0.07 & 100.80 & 0.16 \\
17 & -1.59 & 0.08 &  99.93 & 0.22 \\
19 & -2.43 & 0.16 &  99.67 & 0.19 \\
\hline
\end{tabular}
\captionsetup{justification=raggedright, singlelinecheck=false}
{\caption[Stredná hodnota a smerodajná odchýlka napätia vyrovnaných
    pásov a hrúbky oxidu]{Stredná hodnota a smerodajná odchýlka
    napätia vyrovnaných pásov a hrúbky oxidu.}\label{tab:7.4}}
\end{table}

Z tabuľky \ref{tab:7.4} je vidieť, že hodnoty $\overline V_{fb}$
súvisia so strednými hodnotami hrúbky oxidovej vrstvy $\overline
h_{ox}$, preto sme túto závislosť zobrazili na obrázku \ref{fig:7.6}.

\begin{figure}[h!]\centering
\includegraphics{Figures/fig-7-6.eps}
\captionsetup{justification=raggedright, singlelinecheck=false}
{\caption[Závislosť strednej hodnoty $\overline V_{fb}$ od strednej
    hodnoty hrúbky oxidovej vrstvy $\overline h_{ox}$]{Závislosť
    strednej hodnoty $\overline V_{fb}$ od strednej hodnoty hrúbky
    oxidovej vrstvy $\overline h_{ox}$ pre kremíkové dosky číslo 1, 3,
    5, 7, 9, 11, 13, 15 a 17.}\label{fig:7.6}}
\end{figure}
% OBR28.BIT

Hodnota korelačného koeficientu

$$R(\overline V_{fb} ,\overline h_{ox}) = -0.78$$

súhlasí s teoretickým vzťahom, určujúcim závislosť hodnoty $V_{fb}$ od
veľkosti poruchového náboja v oxidovej vrstve a na rozhraní
$Si-SiO_{2}$ $Q_{dc}$ a od veľkosti kapacity oxidovej vrstvy $C_{ox}$

\begin{equation}\label{eq:7.3}
V_{fb}  = \varphi_{ms} + \frac{Q_{dc}}{C_{ox}}
\end{equation}

kde $\varphi_{ms}$ predstavuje rozdiel výstupných potenciálov medzi
polovodičom a kovom.

Pre koeficienty lineárnej regresie 

$$V_{fb}  = a + b h_{ox}$$                        

sme dostali hodnoty 

$$a = 5.48 \times 10^{-3}  \qquad  b = -1.41 \times 10^{7}$$

Na štyroch kremíkových doskách bola určená hustota pascí rozhrania
$Si-SiO_2$ $D_{it}$ . Ako vidiet ť z tabuľky \ref{tab:7.5}, stredné
hodnoty $\overline D_{it}$ sa pohybujú v oblasti $2.0-5.0\times
10^{14}$, čo hovorí o dobrej kvalite rozhrania $Si-SiO_{2}$.

O kvalite kryštálu podáva informáciu veľkosť generačného času
minoritných nosičov náboja. Aby sme mohli porovnať kvalitu kryštálu
pre jednotlivé dosky, určili sme na každej kremíkovej doske plošné
rozloženie $\tau_{g}$ v hĺbke od $0.9$ do $1.3\mu m$. Pre všetky dosky
sme potom určili strednú hodnotu $\overline\tau_{g}$ a smerodajnú
odchýlku $\delta\tau_{g}$, ktorých hodnoty sú uvedené v tabuĽke
\ref{tab:7.6}. Hodnoty $\overline\tau_{g}$ sa pohybujú v rozmedzí
$0.41 - 2.25 ms$, čo hovorí o vysokej kvalite substrátu. Zároveň z
tabuľky \ref{tab:7.6} vidieť, že hodnoty $\overline \tau_{g}$ sa
pohybujú náhodne a nie je možné nájsť závislosť od ďalších, predtým
spomenutých parametrov.

\begin{table}[h!]\centering
\begin{tabular}{|c|c|c|}
\hline
č. & $D_{it}[m^{-2}eV^{-1}]$ & $\delta D_{it}[m^{-2}eV^{-1}]$ \\ 
\hline
 3 & $4.42 \times 10^{14}$ & $0.25 \times 10^{14}$ \\
 7 & $2.60 \times 10^{14}$ & $0.15 \times 10^{14}$ \\
 9 & $2.74 \times 10^{14}$ & $0.15 \times 10^{14}$ \\
12 & $3.55 \times 10^{14}$ & $0.16 \times 10^{14}$ \\
\hline
\end{tabular}
\captionsetup{justification=raggedright, singlelinecheck=false}
\caption [Stredná hodnota a smerodajná odchýlka hustoty pascí
  rozhrania $Si-SiO_{2}$ v strede zakázaného pásma.] {Stredná hodnota
  a smerodajná odchýlka hustoty pascí rozhrania $Si-SiO_{2}$ v strede
  zakázaného pásma.}
\label{tab:7.5}
\end{table}

\begin{table}[h!]\centering
\begin{tabular}{|c|c|c|}
\hline
č. & $\overline\tau_{g}[ms]$ & $\delta\tau_{g}[ms]$ \\ 
\hline
 1 & 1.93 & 0.12 \\
 3 & 1.48 & 0.09 \\
 5 & 1.84 & 0.09 \\
 7 & 1.67 & 0.10 \\
10 & 1.95 & 0.09 \\
12 & 0.41 & 0.02 \\
15 & 1.74 & 0.09 \\
17 & 2.25 & 0.14 \\
\hline
\end{tabular}
\captionsetup{justification=raggedright, singlelinecheck=false}
{\caption[Stredná hodnota a smerodajná odchýlka generačnej doby
    života minoritných nosičov náboja] {Stredná hodnota a smerodajná
    odchýlka generačnej doby života minoritných nosičov
    náboja.} \label{tab:7.6}}
\end{table}

\begin{thebibliography}{}
\bibitem[7.1]{7.1}
Renyi A.: Teorie pravdepodobnosti. Academia Praha 1972.
\end{thebibliography}
