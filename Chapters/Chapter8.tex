% Chapter 8
\chapter{Súhrn výsledkov s uvedením nových poznatkov.}
\label{Chapter8}
\lhead{Chapter 8. \emph{Súhrn výsledkov s uvedením nových poznatkov}}
% This is for the header on each page - perhaps a shortened title
%----------------------------------------------------------------------

V dizertačnej práci sú uvedené výsledky získané pri skúmaní parametrov
štruktúr MOS s nehomogénnym rozložením prímesí v podpovrchovej oblasti
polovodiča. Vzhľadom k tomu, že u nás nebola doposiaľ táto
problematika komplexne riešená, bolo potrebné zhrnúť súčasné poznatky
z tejto oblasti, zvládnuť metodiku merania a vyhodnotenia parametrov a
prakticky ich realizovať. Praktická stránka pozostávala z realizácie
jednotlivých kapacitných metód a vytvorenia komplexného pracoviska pre
meranie a vyhodnotenie plošného rozloženia parametrov štruktúr MOS s
perspektívou ďalšieho využitia pre skúmanie korelácií medzi
jednotlivými parametrami, prípadne medzi parametrami iných testovacích
štruktúr, alebo technologických postupov planárnej technológie.

Dizertačná práca priniesla následovné výsledky:

\begin{enumerate}

% 1
\item Vyriešili sme jedno-dimenzionálnu Poissonovu rovnicu pre
  nehomogénne dotovaný substrát polovodiča a vypočítali sme teoretické
  CV závislosti štruktúry MOS. Numerické riešenie Poissonovej rovnice
  umožnilo získať informácie o fyzikálnych dejoch v štruktúre MOS v
  procese merania CV závislostí a umožnilo overenie použitých
  aproximácií pri výpočte koncentračných profilov.

%2
\item Okrem Štandardnej vysokofrekvenčnej a kvázistatickej CV
  metódy sme realizovali postup merania štruktúr MOS metódou QC, kde
  bolo potrebné:

\begin{itemize}
\item navrhnúť a realizovať prípravok pre umiestnenie meranej vzorky a
  vzduchového kondenzátora
\item  minimalizovať zvodové prúdy a parazitné kapacity
\item zvoliť vhodné postupy určovania parazitných kapacít
\item vytvoriť programové vybavenie pre zber a spracovanie dát.
\end{itemize}

Ako sa ukázalo počas zavádzania metódy, ako aj počas následovných
experimentov, implementácia metódy QC vyžaduje vytvorenie dôkladného
meracieho pracoviska s dôrazom na minimalizáciu zvodových prúdov a
parazitných kapacít.

%3
\item Pre meranie generačného času života minoritných nosičov náboja
  sme implementovali metódu konštantnej šírky OPN (CCT), ktorá
  umožňuje efektívne vyhodnocovanie kvality polovodičových substrátov
  pre vysoké hodnoty $\tau_{g}$ . Pri tvorbe riadiaceho programu sme
  navrhli a realizovali algoritmus pre udržanie konštantnej
  nerovnovážnej kapacity štruktúry MOS a merania závislosti
  $V_{g}(t)$.

%4
\item Všetky použité metódy (HF, LF, QC a CCT) boli automatizované
  pomocou osobného počítača PC AT s interfejsom PCIIA, pričom bolo
  potrebné zvládnuť riadenie zbernice IMS-2 a efektívne využiť
  autonómne schopnosti použitých prístrojov. Pre skúmanie plošného
  rozloženia parametrov štruktúr MOS na kremíkovej doske sme
  vypracovali programový balík pozostávajúci z približne 40 programov,
  ktoré vykonávajú zber dát, ich spracovanie a určenie parametrov
  štruktúr MOS.  Zároveň sú k dispozícii programy pre zobrazenie
  získaných výsledkov.  Stručný prehľad vytvorených programov, ktoré
  slúžia pre zber, spracovanie a zobrazenie dát plošného rozloženia
  parametrov štruktúr MOS, je v dodatku \ref{app:AppendixH}.

%5
\item Pri výpočte koncentračných profilov dotujúcich prímesí v
  podpovrchovej oblasti polovodiča sme aplikovali:
\begin{itemize}
\item korekciu aproximácie hlbokého ochudobnenia pri povrchu
  polovodiča
\item korekciu vzhľadom na hustotu pascí rozhrania $Si-SiO_{2}$
\item výpočet šírky oblasti OPN pomocou aproximácie priebehu
  elektrického potenciálu v polovodiči
\item výpočet priestorového rozloženia dotujúcich atómov z
  koncentračného profilu majoritných nosičov náboja.
\end{itemize}

%6
\item Vhodnosť použitých aproximácií sme overili riešením Poissonovej
  rovnice a výpočtom koncentračného profilu majoritných nosičov náboja
  z teoretickej CV závislosti.  Tu sa ukázalo, že pre určovanie
  koncentračných profilov implantovaných prímesí testovaných
  kremíkových dosiek v záverečnom experimente je použitie uvedených
  aproximácii vhodné.

%7
\item Pre rôzne dávky implantácie v rozsahu od $0.6 \times 10^{15}$ do
  $60.0 \times 10^{15} m^{-2}$ sme určili na 10 kremíkových doskách
  priebehy koncentračných profilov $N(x)$ na približne 300 štruktúrach
  MOS každej testovanej dosky a vypočítali sme ich stredné
  hodnoty. Zároveň sme určili strednú hodnotu časti dávky
  implantovaných iónov, ktoré sa stali elektricky aktívnymi v
  polovodiči $D_{a}$ . Ukázalo sa, že závislosť medzi dávkou
  implantácie $D_{i}$ a množstvom elektricky aktívnych, implantovaných
  prímesí $D_{a}=f(D_{i})$ je lineárna, čo usudzujeme z hodnoty
  korelačného koeficientu $R(D_{i},D_{a})=0.99$.  Lineárnou regresiou
  závislosti $D_{a}=f(D_{i})$ sme zistili, že z pôvodnej implantovanej
  dávky sa do polovodiča dostalo a stalo sa elektricky aktívnymi
  $71\%$ iónov. Uvedenú metodiku možno využiť pri kontrole procesu
  implantácie a bola vypracovaná na základe požiadavky Tesly Piešťany.

%8
\item Vypočítané koncentračné profily sme overili simuláciou
  technologického procesu implantácie pomocou funkcie Pearson IV. Z
  porovnania výsledkov zistených pomocou kapacitných meraní a
  simulácie vidieť malý rozdiel, ktorý môže byť spôsobený tým, že v
  procese poimplantačného tepelného spracovania neboli všetky
  implantované ióny aktivované.  Zistené rozdiely sú však
  minimálne. Bola vypracovaná metodika na sledovanie plošného
  rozloženia hĺbkových profilov dotujúcich prímesí v rámci hraníc
  použiteľnosti kapacitnej metódy s možnosťou sledovať plošné
  rozloženie v ľubovolnej hĺbke pod povrchom polovodiča. Na základe
  experimentálnych výsledkov možno konštatovať, že proces implantácie
  prebiehal na 4 palcových kremíkových substrátoch reprodukovateľne s
  vysokou homogenitou rozloženia prímesí, čím sme overili kvalitu
  implantačného zariadenia.

%9
\item Predchádzajúce výsledky, získané pri určovaní hĺbkových
  koncentračných profilov, sa využili pri skúmaní vlastností rozhrania
  $Si-SiO_{2}$ štruktúr MOS s implantovaným substrátom. Pre analýzu
  týchto štruktúr bola použitá:
\begin{enumerate}
%a
\item diferenciálna kapacitná metóda, porovnávajúca HF a LF CV
  závislosť štruktúry MOS, pričom výpočet $\varphi_{s}(V_{g})$
  zohľadňuje hĺbkový profil prímesí
%b
\item kvázistatická CV metóda, založená na porovnaní experimentálnej a
  teoretickej CV závislosti.
\end{enumerate}
Z porovnania výsledkov získaných oboma metódami vyplýva, že hustota
pascí rozhrania $Si-SiO_{2}$ v strede zakázaného pásu sa prakticky
nelíši. Obe metodiky možno aplikovať na CV závislosti určené pomocou
HF a kvázistatickej metódy, alebo na CV závislosti štruktúry MOS
určené pomocou QC metódy. Zároveň sme zvládli postup určenia priebehu
povrchového potenciálu $\varphi_s(V_g)$ pomocou QC metódy, alebo
integráciou LF CV závislosti. Pre výpočet hustoty pascí z porovnania
experimentálnej a teoretickej CV závislosti sme použili teoretickú LF
CV závislosť určenú pomocou numerického riešenia Poissonovej rovnice
pre nehomogénne rozloženie dotujúcich prímesí v polovodiči.

%10
\item Z porovnania HF a LF CV závislosti sme určili plošné rozloženie
  hustoty pascí rozhrania $Si-SiO_{2}$ . Stredná hodnota hustoty pascí
  rozhrania v strede zakázaného pásu testovaných dosiek sa pohybovala
  v intervale od $2.6 \times 10^{14}$ do $4.4 \times 10^{14}
  m^{-2}eV^{-1}$. Tieto hodnoty sú na dolnej hranici rozlišovacej
  schopnosti použitej metódy a hovoria o dobrej kvalite rozhrania
  $Si-SiO_{2}$ testovaných vzoriek a súčasne o kvalite opracovania
  polovodičových substrátov.

%11
\item V záverečnom experimente sme metódou CCT určili plošné
  rozloženie strednej hodnoty $\tau_{g}$ na 8 kremíkových doskách v
  hĺbke od $0.9$ do $1.3 mm$, ktoré sa pohybujú od $0.41$ do $2.25
  ms$, čo hovorí o vysokej kvalite substrátu. \newline Výhodou
  použitej metódy je, že vyhodnocuje $\tau_{g}$ z generačného prúdu
  minoritných nosičov náboja len z oblasti OPN a eliminuje vplyv
  generácie minoritných nosičov náboja mimo túto oblasť, čo sa
  prejavilo na reprodukovateľnosti hodnôt tohto parametra. \newline
  Bola zistená vzájomná súvislosť medzi hĺbkovým profilom doby života
  a koncentračným profilom implantovaných prímesí.  Dosiahnuté
  výsledky poukazujú, že u skúmaných vzoriek dominantný mechanizmus,
  ktorý určuje dobu života je rozptyl na ionizovaných prímesiach. To
  taktiež potvrdzuje, že skúmané polovodičové substráty sú
  vysoko-kvalitné z hľadiska defektov a preto je rozptyl na nich
  vzhľadom na rozptyl na ionizovaných prímesiach zanedbateľný.

%12
\item Určili sme plošné rozloženie hrúbky oxidovej vrstvy vypočítanej
  z kapacity štruktúry MOS v akumulácii na 10 kremíkových doskách, z
  ktorého vidieť nehomogenity hrúbky $SiO_{2}$ spôsobené nerovnomerným
  rozložením teploty a turbulenciami oxidačnej atmosféry v trubici
  oxidačnej pece. \newline Stredné hodnoty hrúbky $SiO_{2}$ na jednotlivých
  doskách sa pohybujú v intervale od $94.14$ do $102.85 nm$, pričom
  v technologickom procese výroby štruktúr MOS bola požadovaná hodnota
  $100 nm$. \newline Zároveň je z tabuľky stredných hodnôt vidieť, že
  hrúbka oxidu je najnižšia pre kremíkové dosky, ktoré sa nachádzali
  na prednom a zadnom konci oxidačnej lodičky a najhrubší oxid sa
  vytvoril v strede, čo bolo spôsobené rozdelením teploty v oxidačnej
  trubici. Tieto poznatky sú v súlade s Bermannovým modelom mechanizmu
  termickej oxidácie.

%13
\item Skúmali sme plošné rozloženie napätia vyrovnaných pásov $V_{fb}$
  pre 10 kremíkových dosiek. Stredné hodnoty $V_{fb}$ na jednotlivých
  kremíkových doskách sa pohybujú v rozsahu od $-1.24$ do $-2.43
  V$. Vzhľadom k tomu, že náboj pascí rozhrania je pre skúmané dosky
  malý, rozptyl stredných hodnôt $V_{fb}$ môže byť spôsobený nábojom
  alkalických iónov v oxide.

%14
\item Pre riešenie rovníc matematickej fyziky sme použili metódy
  numerickej matematiky. Riešili sme diferenciálnu rovnicu druhého
  rádu s počiatočnými podmienkami pomocou metódy prediktor-korektor so
  štartovacím úsekom Runge-Kutta, hľadali sme korene nelineárnej
  rovnice metódou dotyčníc, použili sme číslicové filtre pre
  vyhladenie a určenie derivácií experimentálne získaných dát. Použili
  sme funkcie knižnice NAG pre aproximáciu kubickými splajn-funkciami.

%15
\item Vypracovali sme systém a štruktúry dátových súborov, ktoré
  uchovávajú namerané dáta a určené parametre štruktúry MOS, pričom
  sme zohľadnili vzťahy medzi jednotlivými metodikami merania a
  určovania parametrov, čo prispieva k väčšej efektívnosti programov a
  umožňuje ďalšie použitie získaných výsledkov.

\end{enumerate}
