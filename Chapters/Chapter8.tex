% Chapter 8
\chapter{Summary of results with new findings.}\label{Chapter8}
\lhead{Chapter 8. \emph{Summary of results with new findings}}
%----------------------------------------------------------------------

The dissertation presents the results obtained by investigating the
parameters of MOS structures with inhomogeneous distribution of
impurities in the subsurface region of the semiconductor. Since in our
country there has not been so far this problem has been
comprehensively addressed in our country, it was necessary to
summarize the current knowledge in this field, to master the
methodology of measurement and evaluation of parameters and to
implement them practically. The practical side consisted of the
implementation of individual capacity methods and the creation of a
complex workplace for measurement and evaluation of the area
distribution parameters of MOS structures with with a view to further
use for the investigation of correlations between individual
parameters, or between the parameters of other test structures or
technological procedures of planar technology.

The dissertation has produced the following results:

\begin{enumerate}

% 1
\item We have solved the one-dimensional Poisson equation for
  inhomogeneously doped semiconductor substrate and calculated the
  theoretical CV dependences of the MOS\@ structure. Numerical
  solution of the Poisson equation made it possible to obtain
  information about the physical processes in the structure MOS in the
  process of CV dependence measurements and allowed verification of
  the used approximations in the calculation of the concentration
  profiles.

%2
\item In addition to the standard high-frequency and quasi-static CV
  methods method, we implemented a procedure for measuring MOS
  structures by QC method, where the needed:

  \begin{itemize}
  \item design and implement a fixture for the location of the measured sample and
    air condenser
  \item minimise leakage currents and parasitic capacitances
  \item select appropriate procedures for determining parasitic capacitances
  \item to develop software for data acquisition and processing.
  \end{itemize}

  As has been demonstrated during the implementation of the method, as
  well as during subsequent experiments, the implementation of the QC
  method requires the development of a thorough measurement
  workstation with an emphasis on minimizing leakage currents and
  parasitic capacitances.

%3
\item To measure the generation lifetime of minority charge carriers
  we have implemented the constant width OPN (CCT) method, which
  allows efficient evaluation of the quality of semiconductor
  substrates for high values of $\tau_{g}$. In the design of the
  control program, we designed and implemented an algorithm to
  maintain constant non-equilibrium capacitance of the MOS structure
  and to measure the dependence $V_{g}(t)$.

%4
\item All methods used (HF, LF, QC and CCT) were automated using a PC
  AT personal computer with a PCIIA interface, while the it was
  necessary to master the IMS-2 bus control and to make efficient use
  of the autonomous capabilities of the instruments used. To
  investigate the area distribution of the parameters of the MOS
  structures on the silicon wafer, we developed a software package
  consisting of approximately 40 programs, that perform data
  acquisition, processing and parameter determination of MOS
  structures\@ At the same time, programs are available for displaying
  the results obtained. A brief overview of the developed programs,
  which are used for the collection, processing and display of areal
  distribution data parameters of MOS structures is given in
  Appendix~\ref{app:AppendixH}.

%5
\item When calculating the concentration profiles of the dopant
  impurities in the subsurface region of the semiconductor, we
  applied:

  \begin{itemize}
  \item correction to the deep depletion approximation at the surface
    semiconductor
  \item correction with respect to the trap density of the $Si-SiO_{2}$ interface
  \item calculation of the OPN region width using the waveform approximation
    of the electric potential in the semiconductor
  \item calculation of the spatial distribution of dopant atoms from
    the concentration profile of the major charge carriers.
  \end{itemize}

%6
\item We have verified the suitability of the approximations used by
  solving the Poisson equation and calculating the concentration
  profile of the major charge carriers from the theoretical CV
  dependence. Here it was shown that for the determination of the
  concentration profiles of the implanted impurities tested silicon
  wafers in the final experiment is the use of the above approximation
  is appropriate.

%7
\item For different implantation doses ranging from $0.6 \times
  10^{15}$ to $60.0 \times 10^{15} m^{-2}$ we determined on 10 silicon
  wafers $N(x)$ concentration profiles on approximately 300 structures
  MOS of each tested wafer and calculated their mean values. At the
  same time, we determined the mean value of the dose fraction of the
  implanted ions that became electrically active in $D_{a}$
  semiconductor. It turned out that the dependence between the dose
  $D_{i}$ implantation and the amount of electrically active,
  implanted $D_{a}=f(D_{i})$ is linear, as inferred from the value of
  of the correlation coefficient $R(D_{i},D_{a})=0.99$.  Linear
  regression of the dependence $D_{a}=f(D_{i})$, we found that from
  the original implanted dose entered the semiconductor and became
  electrically active $71\%$ of the ions. The above methodology can be
  used to control the process implantation and was developed based on
  the request of Tesla Piešťany.

%8
\item The calculated concentration profiles were verified by
  simulation technological process of implantation using Pearson IV
  function. Z comparison of the results obtained by capacitance
  measurements and simulation show a small difference, which may be
  due to the fact that in the post-implantation heat treatment process
  were not all implanted ions were activated. However, the differences
  found are minimal. A methodology has been developed to track the
  area distribution of the depth profiles of the dopant impurities
  within the boundaries of the of applicability of the capacitance
  method with the ability to track the areal distribution at arbitrary
  depths below the semiconductor surface. Based on experimental
  results, it can be concluded that the implantation process was
  reproducibly performed on 4 inch silicon substrates with high
  homogeneity of the impurity distribution, thus verifying the quality
  of of the implantation device.


%9
\item Previous results, obtained in the determination of depth
  concentration profiles have been used to investigate the properties
  of the interface $Si-SiO_{2}$ MOS structures with implanted
  substrate. For the analysis of these structures was used:

  \begin{enumerate}
  \item differential capacitance method, comparing HF and LF CV
    dependence of the MOS structure, calculating $\varphi_{s} (V_{g})$
    takes into account the depth profile of the impurities
  \item quasi-static CV method, based on comparison of experimental
    and theoretical CV dependence.
  \end{enumerate}

  A comparison of the results obtained by the two methods shows that
  the density of the $Si-SiO_{2}$ interface traps in the center of the
  forbidden band is practically does not differ. Both methodologies
  can be applied to the CV dependences determined by HF and
  quasi-static methods, or to the CV dependence of the MOS structure
  determined using the QC method. At the same time, we have mastered
  the procedure of determining the waveform surface potential
  $\varphi_s (V_g)$ using the QC method, or by integration of the LF
  CV dependence. To calculate the trap densities from the comparison
  experimental and theoretical CV dependence, we used the theoretical
  LF CV dependence determined by numerical solution of the Poisson
  equation for an inhomogeneous distribution of dopant impurities in
  the semiconductor.

%10
\item From the comparison of HF and LF CV dependence we determined the
  area distribution of the $Si-SiO_{2}$ interface trap densities. The
  mean value of trap density of the interface traps in the middle of
  the forbidden band of the tested plates ranged ranging from
  $2.6\ \times\ 10^{14}$ to $4.4\ \times\ 10^{14}m^{-2}eV^{-1}$. These
  values are at the lower limit of the resolving of the method used
  and are indicative of good interface quality $Si-SiO_{2}$ of the
  tested samples and at the same time the quality of the processing
  semiconductor substrates.

%11
\item In the final experiment, we used the CCT method to determine the
  area distribution of the mean value of $\tau_{g}$ on 8 silicon
  wafers in $0.9$ to $1.3 mm$ in depth, ranging from $0.41$ to $2.25
  ms$, indicating high substrate quality.
  \newline Advantage of the method used is that it evaluates
  $\tau_{g}$ from the generation current of minority charge carriers
  from the OPN region only and eliminates the influence of of the
  generation of minority charge carriers outside this region, which
  the reproducibility of the values of this parameter.
  \newline A correlation was found between the depth profile of the
  lifetime and the concentration profile of implanted
  admixtures. Achieved results indicate that the dominant mechanism in
  the studied samples, that determines the lifetime is the dispersion
  on ionised impurities. This also confirms that the semiconductor
  substrates investigated are high quality in terms of defects and
  therefore the scattering on them with respect to the scattering on
  ionised impurities is negligible.

%12
\item We have determined the areal distribution of the oxide layer
  thickness calculated from the capacity of the MOS structure in
  accumulation on 10 silicon wafers, from which shows inhomogeneities
  in the thickness of $SiO_{2}$ caused by uneven temperature
  distribution and turbulence of the oxidizing atmosphere in the tube
  of the oxidation furnace. \newline Mean values of $SiO_{2}$
  thickness on individual plates range from $94.14$ to $102.85 nm$,
  while in the technological process of manufacturing MOS structures
  was the required value of $100 nm$. \newline At the same time, the
  table of mean values shows that the oxide thickness is the lowest
  for silicon wafers, which were located at the front and back ends of
  the oxidation boat and the thickest oxide was formed in the middle,
  which was due to the splitting temperature in the oxidation
  tube. These findings are consistent with Bermann's model of the
  thermal oxidation mechanism.

%13
\item We have investigated the areal distribution of the stresses of
  the aligned $V_{fb}$ strips for 10 silicon wafers. The mean values
  of $V_{fb}$ on individual silicon wafers range from $-1.24$ to
  $-2.43 V$. Since the interface trapping charge for the studied
  plates is small, the scatter in the mean values of $V_{fb}$ may be
  due to the charge alkaline ions in the oxide.

%14
\item To solve the mathematical physics equations, we used the methods
numerical mathematics. We solved the differential equation of the
second order with initial conditions using the predictor-corrector
method with Runge-Kutta starting term, we searched for the roots of
the nonlinear equation by the tangent method, we used numerical
filters for smoothing and determining the derivatives of the
experimentally obtained data. They used NAG library functions for
approximation by cubic spline functions.

%15
\item We developed a system and data file structures that store the
  measured data and the determined parameters of the MOS structure,
  while we have taken into account the relationships between the
  different measurement methodologies and of parameter determination,
  contributing to greater efficiency of the programs and allowing
  further use of the obtained results.

\end{enumerate}
