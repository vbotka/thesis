% Chapter 2

\chapter{Ciele dizertačnej práce.}% Main chapter title
\label{Chapter2} % For referencing the chapter elsewhere, use \ref{Chapter1} 
\lhead{Chapter 2. \emph{Ciele dizertačnej práce}} % This is for the header on each page - perhaps a shortened title

%----------------------------------------------------------------------------------------
\begin{enumerate}
\item Vybudovanie automatizovaného experimentálneho pracoviska pre
  analýzu elektrofyzikálnych vlastností štruktúr MOS s nehomogénnym
  rozložením dotujúcich prímesí v substráte, s možnosťou sledovať
  plošné rozloženie skúmaných parametrov na kremíkovom substráte.
  Súčasťou pracoviska sú vysokofrekvenčná C-V metóda (vrátane
  ochudobnenej C-V metódy), nízkofrekvenčná C-V metóda, metóda
  konštantnej šírky OPN (CCT) a Q-C metóda.
\item Metódy uvedené v bode 1\@. automatizovať s využitím riadiaceho
  počítača PC AT pod operačným systémom MS DOS\@. Pre realizáciu metód
  využiť interfejs GPIB PCIIA, meracie prístroje HP4280a, Keithley 642
  a hrotové krokovacie zariadenie Zond A5.
\item Realizované metódy použiť pre určenie hĺbkových koncentračných
  profilov implantovaných prímesí, hrúbky oxidovej vrstvy, napätia
  vyrovnaných pásov, hustoty pascí rozhrania Si-SiO a hĺbkového
  profilu generačného času života minoritných nosičov náboja.
  Analyzovať problémy vznikajúce pri určovaní parametrov štruktúr MOS
  v spojení s nehomogénnou dotáciou substrátu.
\item Zistiť hĺbkové koncentračné profily aktívnych prímesí a ich
  rozloženia na kremíkovom substráte pre rôzne dávky implantácie v
  rozsahu od $0.6\times{10}^{15}$ do
  $60.0\times{10}^{15}{m}^{-2}$. Zistiť ako vplýva implantačná dávka
  na vlastnosti rozhrania Si-SiO a hĺbkový profil času života
  minoritných nosičov náboja.  Navrhnúť metodiku pre identifikáciu
  množstva implantovaných iónov v polovodičovom substráte pomocou
  kapacitnej metódy.
\end{enumerate}
