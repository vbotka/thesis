% Chapter 9
\chapter{Závery pre prax a rozvoj vednej disciplíny.}\label{Chapter9}
\lhead{Kapitola 9. \emph{Závery pre prax a rozvoj vednej disciplíny}}

Z hľadiska zvolených cieľov bolo dosiahnutých viacero poznatkov, ktoré
sa uplatnili v praxi pri kontrole technologických postupov vytvárania
polovodičových štruktúr planárnou technológiou. Záverom môžeme prínosy
práce zhrnúť do nasledovných bodov:

\begin{enumerate}

% 1
\item Realizácia komplexného automatizovaného pracoviska pre skúmanie
  elektro-fyzikálnych vlastností štruktúr MOS s nehomogénnym
  rozložením prímesí s možnosťou sledovania plošného rozloženia:

  \begin{itemize}
  \item koncentračného profilu dotujúcich prímesi $N(x)$ pre rôzne
    hĺbky $x$
  \item hĺbkového profilu času života $\tau_{g}(x)$ pre rôzne hĺbky
    $x$
  \item hustoty pascí rozhrania $Si-SiO_{2}$ $D_{it}(E_{c}-E)$ pre
    rôzne energie v zakázanom páse polovodiča
  \item napätia vyrovnaných pásov $V_{fb}$
  \item hrúbky oxidovej vrstvy $h_{ox}$.
  \end{itemize}

% 2
\item Výber vhodných numerických metód a ich použitie pre riešenie:

  \begin{itemize}
  \item jedno-dimenzionálnej Poissonovej rovnice
  \item nelineárnej rovnice pre určenie povrchového potenciálu z
    kapacity OPN $C_{sc}$
  \item vyhladenia a interpolácie experimentálne určených dát
  \item výpočet derivácie experimentálne určených dát.
  \end{itemize}

% 3
\item Vytvorenie programového vybavenia pre riadenie experimentálnych
  meraní, spracovanie a zobrazenie výsledkov vybraných parametrov
  štruktúr MOS s nehomogénnym rozložením prímesí v podpovrchovej
  oblasti polovodiča a ich plošného rozloženia.

% 4
\item Skúmanie homogenity procesu implantácie na 4-palcových
  kremíkových doskách s rozsahom dávok od $0.6 \times 10^{14}$ do
  $60.0 \times 10^{14} m^{-2}$ vzhľadom na:

  \begin{itemize}
  \item hĺbkový profil aktívnych prímesí
  \item vlastnosti rozhrania $Si-SiO_{2}$
  \item hĺbkový profil generačného času života minoritných nosičov
    náboja.
  \end{itemize}

% 5
\item Navrhla a realizovala sa metodika pre určenie implantovanej
  dávky prímesí. Experimentálne výsledky boli overené simuláciou
  technologického procesu pomocou funkcie Pearson IV\@. Porovnanie
  experimentálnych a teoretických výsledkov vykazujú minimálny
  rozdiel. Navrhnutá metodika kontroly implantovanej dávky je
  aplikovateľná v praxi.

% 6
\item Zistila sa vzájomná súvislosť medzi koncentračným profilom
  dotujúcich prímesí a hĺbkovým profilom generačného času života
  minoritných nosičov náboja. Profil času života kvalitných
  kremíkových substrátov nie je určený mechanizmom rozptylu na
  náhodných defektoch substrátu, ale len na implantovaných prímesiach.

\end{enumerate}
