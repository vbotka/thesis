% Introduction
\chapter{Introduction.} % Main chapter title
\label{Introduction} % For referencing the chapter elsewhere, use \ref{Chapter1} 
\lhead{\emph{Introduction}} % This is for the header on each page - perhaps a shortened title
%----------------------------------------------------------------------------------------

\par Oblasť diagnostiky štruktúr MOS bola v posledných desaťročiach
predmetom rozsiahleho výskumu a v súčasnosti sa nachádza v štádiu
rutinného používania. Mnohé metodiky určovania parametrov štruktúr MOS
možno v súčasnosti považovať za uzavreté o čom svedčia rozsiahle
monografie publikované v tejto oblasti \cite{I.1} \cite{I.2} \cite{I.3}
\cite{I.4}. Zároveň je k dispozícii profesionálne prístrojové vybavenie
pre určovanie parametrov štruktúr MOS, ktoré slúži pre rýchlu
diagnostiku technológie výroby polovodičových prvkov a integrovaných
obvodov. Mnohé technologické postupy používané v súčasnosti pri výrobe
diskrétnych polovodičových súčiastok a integrovaných obvodov na báze
kremíka sú dostatočne preskúmané a pri ich používaní sa dosahuje
vysoká reprodukovateľnosť parametrov, avšak stále ešte možno nájsť
oblasti, v ktorých vývoj diagnostiky polovodičových prvkov pomocou
štruktúr MOS nie je ukončený.

\par V snahe o vyššiu efektívnosť výroby sa prejavuje tendencia
používať kremíkové dosky stále väčších priemerov, čo prináša v spojení
s problematikou výťažnosti potrebu štatistického prístupu k
vyhodnocovaniu testovaných parametrov jednotlivých štruktúr. Merané
parametre testovacích štruktúr úzko súvisia s hodnotami
technologických parametrov dosiahnutých v procese výroby a funkčnosť
hotovej súčiastky závisí od veľkého množstva jednotlivých
technologických krokov, ktoré sa pohybujú v určitých tolerančných
medziach. Pri zvyšovaní hustoty integrácie a s tým súvisiacim
zmenšovaním rozmerov jednotlivých integrovaných elementov sa stáva
dôležitou otázka tolerančných intervalov technologických
parametrov. Automatizácia výroby súvisí spravidla s veľkým objemom
výroby, ktorej efektívnosť zaisťuje v najväčšej miere jej
technológia. Podrobná analýza technologických operácií nie je zďaleka
jednoduchou záležitosťou a vyžaduje veľké množstvo experimentov a
pokusov. Výsledky týchto experimentov je nutné efektívne snímať a pre
vyhodnotenie pouzit výpočtovú techniku, pričom objem snímaných a
vyhodnocovaných dát vyžaduje použitie databázových systémov.  Existuje
celá rada matematických metód, ktoré sú pre tieto účely vhodné a sú k
dispozícii balíky programov, ktoré riešia problémy analýzy
technologických procesov \cite{I.5}. Pravdepodobne prevláda pri výrobe
integrovaných obvodov heuristický prístup k riešeniu otázok funkčnosti
produktov a objasnenie závislosti niektorých parametrov môže viesť k
exaktnejšiemu rozhodovaniu pri riadení chodu technologických
zariadení.

\par V tejto práci sme sa pokúsili o sprístupnenie informacií o
plošnom rozložení niektorých parametrov štruktúr MOS na kremíkovej
doske so zameraním sa na proces iónovej implantácie, ktorá vytváraním
nehomogénneho hĺbkového rozloženia prímesných atómov v polovodiči
vyžaduje prispôsobenie metód merania tejto podmienke. Plošné
zobrazenie parametrov štruktúr MOS je založené na veľkom množstve
meraní a spracovaní dát a aplikácia kapacitných metód na testovanie
celej kremíkovej dosky si vyžaduje vybudovanie adekvátnych
prostriedkov zberu, spracovania a zobrazenia dát. Niektoré zo
skúmaných parametrov vykazujú geometrickú symetriu a iné sa náhodne
pohybujú v určitom intervale hodnôt. Plošné zobrazenie skúmaných
parametrov tak poskytuje rýchlu a prehľadnú informáciu o kvalite
jednotlivých technologických krokov, od ktorých hodnota skúmaného
parametra závisí. Zároveň si možno pomocou tejto vizuálnej informácie
vytvoriť predstavu o fluktuáciách skúmaných parametrov na kremíkovej
doske a neprikladať váhu náhodne zmeraným extrémnym hodnotám.

\par Ďalšímm pokračovanímm predkladanej práce by malo byť hľadanie
súvislostí medzi parametrami testovacích štruktúr rôznych druhov
(napr.štruktúra MOS, tranzistorové štruktúry) so zameraním na
zlepšenie výťažnosti výroby integrovaných obvodov. Pravdepodobne by
bolo možné na základe poznatkov o miere závislosti medzi jednotlivými
parametrami rozhodnúť, aká zmena technologického parametra ovplyvní
funkčnosť produkovanej súčiastky.

\par
Vlastná práca zahŕňa v sebe viacero oblastí vedecko-výskumnej
činnosti, ktoré možno následovne vyčleniť:
\begin{itemize}
\item{fyzikálne základy použitých metód a interpretácia výsledkov}
\item{automatizácia experimentu}
\item{použitie numerickej  matematiky pre riešenie fyzikálnych rovníc a spracovanie dát}
\item{tvorba programového vybavenia} .
\end{itemize}

\par Kapitola 1 obsahuje prehľad súčasného stavu skúmanej problematiky
a zaoberá sa problematikou ideálnej a reálnej štruktúry MOS. Pre
získanie informácií o fyzikálnych dejoch v ideálnej štruktúre MOS sme
numericky vyriešili jednodimenzionálnu Poissonovu rovnicu pre
nehomogénne rozloženie prímesí v polovodiči a zároveň sme získali
teoretické kapacitne-napäťové závislosti štruktúry MOS. Kapitola 2
obsahuje ciele dizertačnej práce. Použité metódy merania a postupy
určovania niektorých parametrov štruktúr MOS sú popísané v kapitole 3
a 4. Pritom boli rozpracované jednako metody používané v minulosti na
Oddelení polovodičových štruktúr a integrovaných obvodov, a zároveň sa
zaviedli nové, doteraz u nás nepužívané metódy (metoda Q-C, metóda
konštantnej šírky oblasti priestorového náboja). Kapitola 5 obsahuje
niektoré myšlienky realizácie automatizovaného zberu a predspracovania
dát. Riešia sa tu problémy štruktúr datových súborov, zabezpečenie
proti strate nameraných dát a manipulácie s jednotlivými záznamami
datových súborov. Zároveň sú v tejto časti spomenuté niektoré riešenia
problémov spojených s automatizáciou experimentu. Napriek tomu, že
tento okruh otázok nesúvisí priamo s fyzikálnou stránkou prevedených
meraní, je potrebné zdôraznit, že bez ich systematického vyriešenia by
prakticky nebolo možné efektívne realizovať automatizované pracovisko
pre sledovanie plošného rozloženia elektrofyzikálnych parametrov
štruktúr MOS. Spracovanie nameraných dát a zobrazenie výsledkov
plošného rozloženia parametrov je uvedené v kapitole 6 a kapitola 7
obsahuje tabuľky s výsledkami experimentu. Zároveň sú tu uvedené aj
výsledky skúmania vzájomnej korelácie niektorých parametrov štruktúr
MOS.

\par Použité programy sú väčšinou napísané v jazyku C a iba niektoré
podprogramy spracovania dat a numerického riešenia fyzikálnych rovnic
používajú programovací jazyk Fortran. Celý systém programov bol
realizovaný pod operačným systémom MS DOS. Prenos programov pod iný
operačný systém by vyžadoval vyriešenie následovných okruhov
problémov:
\begin{itemize}
\item{riadenie zbernice IMS-2}
\item{prisppsôbenie sa systému práce so súbormi}
\item{grafické zobrazenie údajov} .
\end{itemize}

\par Pre väčšiu prehľadnosť textu boli niektoré jej dielčie časti
presunuté do dodatkov.  Práca obsahuje značný počet obrázkov, ktoré
pomáhajú doplniť text a samy o sebe poskytujú množstvo informácií.


\begin{thebibliography}{}
\bibitem[I.1]{I.1}
Nicollian E.H., Brews J.R. : MOS Physics  and  Technology. John Wiley and Sons. New York 1982.
\bibitem[I.2]{I.2}
Grove A.S. : Physics and Technology of Semiconductor devices. John Wiley and Sons. New York 1967.
\bibitem[I.3]{I.3}
Sze S.M. : Physics of semiconductor devices. John Wiley and Sons. New York 1969.
\bibitem[I.4]{I.4}
Runyan W.R., Bean K.E. : Semiconductor integrated  circuit  processing technology. Addison-Wesley 1990.
\bibitem[I.5]{I.5}
AIP ve vývoji technologie pro automatizovanou výrobu. Zborník zo seminara. Dom techniky CSVTS Pardubice 1990.
\end{thebibliography}
