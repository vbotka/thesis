% Introduction
\chapter{Introduction.} % Main chapter title
\label{Introduction} % For referencing the chapter elsewhere, use \ref{Chapter1} 
\lhead{\emph{Introduction}} % This is for the header on each page - perhaps a shortened title
%----------------------------------------------------------------------------------------

\iffalse \par Oblasť diagnostiky štruktúr MOS bola v posledných
desaťročiach predmetom rozsiahleho výskumu a v súčasnosti sa nachádza
v štádiu rutinného používania. Mnohé metodiky určovania parametrov
štruktúr MOS možno v súčasnosti považovať za uzavreté o čom svedčia
rozsiahle monografie publikované v tejto oblasti \cite{I.1} \cite{I.2}
\cite{I.3} \cite{I.4}. Zároveň je k dispozícii profesionálne
prístrojové vybavenie pre určovanie parametrov štruktúr MOS, ktoré
slúži pre rýchlu diagnostiku technológie výroby polovodičových prvkov
a integrovaných obvodov. Mnohé technologické postupy používané v
súčasnosti pri výrobe diskrétnych polovodičových súčiastok a
integrovaných obvodov na báze kremíka sú dostatočne preskúmané a pri
ich používaní sa dosahuje vysoká reprodukovateľnosť parametrov, avšak
stále ešte možno nájsť oblasti, v ktorých vývoj diagnostiky
polovodičových prvkov pomocou štruktúr MOS nie je ukončený.
\fi
Diagnostics of MOS structure has been subject of extensive research in
recent decades and now is at the stage-routine use. At present many
methods of determining the parameters of MOS structure can be
considered as closed as evidenced by the extensive monographies
published in this area, \cite{I.1} \cite{I.2} \cite{I.3}
\cite{I.4}. At the same time the professional instrumentation for the
determination of the parameters of MOS structure is available, which
is used for rapid diagnostics of the technology of semiconductor
devices and integrated circuits. Many technological processes
currently used in the production of discrete semiconductor devices and
integrated circuits based on silicon are well investigated, and high
reproducibility of parameters is achieved, however, there are still
areas where improvements in diagnostics of semiconductor devices using
MOS structures is not completed.

\iffalse
\par V snahe o vyššiu efektívnosť výroby sa prejavuje tendencia
používať kremíkové dosky stále väčších priemerov, čo prináša v spojení
s problematikou výťažnosti potrebu štatistického prístupu k
vyhodnocovaniu testovaných parametrov jednotlivých štruktúr. Merané
parametre testovacích štruktúr úzko súvisia s hodnotami
technologických parametrov dosiahnutých v procese výroby a funkčnosť
hotovej súčiastky závisí od veľkého množstva jednotlivých
technologických krokov, ktoré sa pohybujú v určitých tolerančných
medziach. Pri zvyšovaní hustoty integrácie a s tým súvisiacim
zmenšovaním rozmerov jednotlivých integrovaných elementov sa stáva
dôležitou otázka tolerančných intervalov technologických
parametrov. Automatizácia výroby súvisí spravidla s veľkým objemom
výroby, ktorej efektívnosť zaisťuje v najväčšej miere jej
technológia. Podrobná analýza technologických operácií nie je zďaleka
jednoduchou záležitosťou a vyžaduje veľké množstvo experimentov a
pokusov. Výsledky týchto experimentov je nutné efektívne snímať a pre
vyhodnotenie pouzit výpočtovú techniku, pričom objem snímaných a
vyhodnocovaných dát vyžaduje použitie databázových systémov.  Existuje
celá rada matematických metód, ktoré sú pre tieto účely vhodné a sú k
dispozícii balíky programov, ktoré riešia problémy analýzy
technologických procesov \cite{I.5}. Pravdepodobne prevláda pri výrobe
integrovaných obvodov heuristický prístup k riešeniu otázok funkčnosti
produktov a objasnenie závislosti niektorých parametrov môže viesť k
exaktnejšiemu rozhodovaniu pri riadení chodu technologických
zariadení.
\fi
Aiming at enhancing the production efficiency there is still a
tendency to use silicon wafers of larger diameters. In connection with
the issue of yield it brings the requirement of statistical approach
to evaluation of tested parameters of individual structures. Measured
parameters of test structures are closely related to values of
technological parameters achieved in the production and the
functionality of finished devices depends on a number of individual
technological steps that vary in certain tolerance limits. Tolerance
intervals of technology parameters become important issue with
increasing density of the integration and related decreasing of the
size of individual integrated elements. Automation of production is
typically related to high volume of the production, efficiency of
which is ensured in its greatest extent by the technology. A detailed
analysis of technological operations is far from simple and requires
large volume of experiments and attempts. It is necessary to
efficiently record the results of these experiments and use computers
for calculations, where the volume of recorded and calculated data
requires the use of database systems. There are number of mathematical
methods which are suitable for this purpose and available program
packages that address the analysis of technological processes
\cite{I.5}. Heuristic approach to addressing functionality of the
products probably prevails in the production of integrated
circuits. Clarification of dependencies of some parameters may lead to
more exact management decisions by controlling the technology devices.

\iffalse
\par V tejto práci sme sa pokúsili o sprístupnenie informacií o
plošnom rozložení niektorých parametrov štruktúr MOS na kremíkovej
doske so zameraním sa na proces iónovej implantácie, ktorá vytváraním
nehomogénneho hĺbkového rozloženia prímesných atómov v polovodiči
vyžaduje prispôsobenie metód merania tejto podmienke. Plošné
zobrazenie parametrov štruktúr MOS je založené na veľkom množstve
meraní a spracovaní dát a aplikácia kapacitných metód na testovanie
celej kremíkovej dosky si vyžaduje vybudovanie adekvátnych
prostriedkov zberu, spracovania a zobrazenia dát. Niektoré zo
skúmaných parametrov vykazujú geometrickú symetriu a iné sa náhodne
pohybujú v určitom intervale hodnôt. Plošné zobrazenie skúmaných
parametrov tak poskytuje rýchlu a prehľadnú informáciu o kvalite
jednotlivých technologických krokov, od ktorých hodnota skúmaného
parametra závisí. Zároveň si možno pomocou tejto vizuálnej informácie
vytvoriť predstavu o fluktuáciách skúmaných parametrov na kremíkovej
doske a neprikladať váhu náhodne zmeraným extrémnym hodnotám.
\fi
\par In this work, we tried to make available the information on spatial
distribution of some parameters of MOS structures on silicon wafer
focusing on the ion implantation, that by creating inhomogeneous depth
distribution of the dopant atoms in the semiconductor requires
adapting the measurement methods to this condition. Planar display of
the MOS structure parameters is based on a large number of
measurements and data processing, and the application of the
capacitance methods for testing the entire silicon wafer requires
building adequate means for collecting, processing and displaying of
data. Some of the examined parameters exhibit geometrical symmetry and
other randomly moving in a certain range of values. Thus planar
display of the examined parameters provides rapid and transparent
information about the quality of individual technological steps, the
value of which examined parameter depends. At the same time it is
possible, by means of this visual information, to create an overview
of the fluctuations of the parameters examined on silicon wafer and
disregard randomly measured extreme values.

\iffalse
\par Ďalšímm pokračovanímm predkladanej práce by malo byť hľadanie
súvislostí medzi parametrami testovacích štruktúr rôznych druhov
(napr.štruktúra MOS, tranzistorové štruktúry) so zameraním na
zlepšenie výťažnosti výroby integrovaných obvodov. Pravdepodobne by
bolo možné na základe poznatkov o miere závislosti medzi jednotlivými
parametrami rozhodnúť, aká zmena technologického parametra ovplyvní
funkčnosť produkovanej súčiastky.
\fi
\par The further continuation of this study should be searching
relationships between the parameters of the test structures of
different types (e.g. MOS structure, transistor structures) with a
focus on improving the yield of the production of integrated
circuits. Probably it would possible, in the light of knowledge of the
degree of dependence between parameters, to determine what change of a
technological parameter affects the functionality of the produced
devices.

\iffalse
\par
Vlastná práca zahŕňa v sebe viacero oblastí vedecko-výskumnej
činnosti, ktoré možno následovne vyčleniť:
\begin{itemize}
\item{fyzikálne základy použitých metód a interpretácia výsledkov}
\item{automatizácia experimentu}
\item{použitie numerickej  matematiky pre riešenie fyzikálnych rovníc a spracovanie dát}
\item{tvorba programového vybavenia} .
\end{itemize}
\fi
\par This work comprises a number of areas of scientific research that
can be listed as follows:
\begin{itemize}
\item{basics of used physical methods and interpretation of results}
\item{automation of the experiment}
\item{use of numerical mathematics for solving physical equations and data processing}
\item{programming}.
\end{itemize}

\iffalse \par Kapitola 1 obsahuje prehľad súčasného stavu skúmanej
problematiky a zaoberá sa problematikou ideálnej a reálnej štruktúry
MOS. Pre získanie informácií o fyzikálnych dejoch v ideálnej štruktúre
MOS sme numericky vyriešili jednodimenzionálnu Poissonovu rovnicu pre
nehomogénne rozloženie prímesí v polovodiči a zároveň sme získali
teoretické kapacitne-napäťové závislosti štruktúry MOS. Kapitola 2
obsahuje ciele dizertačnej práce. Použité metódy merania a postupy
určovania niektorých parametrov štruktúr MOS sú popísané v kapitole 3
a 4. Pritom boli rozpracované jednako metody používané v minulosti na
Oddelení polovodičových štruktúr a integrovaných obvodov, a zároveň sa
zaviedli nové, doteraz u nás nepužívané metódy (metoda Q-C, metóda
konštantnej šírky oblasti priestorového náboja). Kapitola 5 obsahuje
niektoré myšlienky realizácie automatizovaného zberu a predspracovania
dát. Riešia sa tu problémy štruktúr datových súborov, zabezpečenie
proti strate nameraných dát a manipulácie s jednotlivými záznamami
datových súborov. Zároveň sú v tejto časti spomenuté niektoré riešenia
problémov spojených s automatizáciou experimentu. Napriek tomu, že
tento okruh otázok nesúvisí priamo s fyzikálnou stránkou prevedených
meraní, je potrebné zdôraznit, že bez ich systematického vyriešenia by
prakticky nebolo možné efektívne realizovať automatizované pracovisko
pre sledovanie plošného rozloženia elektrofyzikálnych parametrov
štruktúr MOS. Spracovanie nameraných dát a zobrazenie výsledkov
plošného rozloženia parametrov je uvedené v kapitole 6 a kapitola 7
obsahuje tabuľky s výsledkami experimentu. Zároveň sú tu uvedené aj
výsledky skúmania vzájomnej korelácie niektorých parametrov štruktúr
MOS.
\fi
\par Chapter 1 provides an overview of the current state of
research problems and deals with the ideal and real MOS structure. To
obtain information on the physical processes in an ideal MOS structure
we numerically solved the one-dimensional Poisson's equation for
inhomogeneous distribution of impurities in semiconductor and at the
same time we have gained the theoretical capacity-voltage curves of
MOS structure. Chapter 2 comprises objectives of this
thesis. Measurement methods and procedures to determine certain
parameters of MOS structures are described in Chapter 3 and 4. In
doing so, the method used in the past on Department of semiconductor
integrated circuits and structures were developed, and new methods,
yet not used on Department, introduced (Q-C method, method CCT
constant width of space charge). Chapter 5 contains some idea of
implementation of automated data collection and
pre-processing. Problems of that data structures, data files,
protection of measured data against loss and manipulation of
individual records are solved.  At the same time some solutions of the
problems related to the automation of the experiment are also
mentioned in this section. Despite the fact that this range of issues
is not directly related to the physical aspect of the measurements, it
must be emphasized that without systematic solution it would be
practically impossible to implement efficient automated workplace
tracking spatial distribution of electro-physical parameters of MOS
structures. Processing of measured data and displaying the results of
the spatial distribution of the parameters is in Chapter 6 and Chapter
7 contains a table with the results of the experiment. At the same
time the results of the examination of cross-correlation of certain
parameters of the MOS structure are also shown.

\iffalse
\par Použité programy sú väčšinou napísané v jazyku C a iba niektoré
podprogramy spracovania dat a numerického riešenia fyzikálnych rovnic
používajú programovací jazyk Fortran. Celý systém programov bol
realizovaný pod operačným systémom MS DOS. Prenos programov pod iný
operačný systém by vyžadoval vyriešenie následovných okruhov
problémov:
\begin{itemize}
\item{riadenie zbernice IMS-2}
\item{prisppsôbenie sa systému práce so súbormi}
\item{grafické zobrazenie údajov} .
\end{itemize}
\fi
\par Programs are written mainly in C, and only some subroutines, to
process data and to solve physical equations numerically, use the
programming language Fortran. The whole system of programs was
developed under the operating system MS DOS. Transfer of the programs
to other operating system would require solution of the following
problems:
\begin{itemize}
\item{IMS-2 (IEEE-488) bus control}
\item{porting of the file processing}
\item{handling of graphical images} .
\end{itemize}

\iffalse \par Pre väčšiu prehľadnosť textu boli niektoré jej dielčie
časti presunuté do dodatkov.  Práca obsahuje značný počet obrázkov,
ktoré pomáhajú doplniť text a samy o sebe poskytujú množstvo
informácií.
\fi
\par To improve clarity of the text some sub-sections were moved to
the amendments. The work contains a large number of images which help
to complete the text and themselves provide large amount of
information.

\begin{thebibliography}{}
\bibitem[I.1]{I.1}
Nicollian E.H., Brews J.R. : MOS Physics  and  Technology. John Wiley and Sons. New York 1982.
\bibitem[I.2]{I.2}
Grove A.S. : Physics and Technology of Semiconductor devices. John Wiley and Sons. New York 1967.
\bibitem[I.3]{I.3}
Sze S.M. : Physics of semiconductor devices. John Wiley and Sons. New York 1969.
\bibitem[I.4]{I.4}
Runyan W.R., Bean K.E. : Semiconductor integrated  circuit  processing technology. Addison-Wesley 1990.
\bibitem[I.5]{I.5}
AIP ve vývoji technologie pro automatizovanou výrobu. Zborník zo seminara. Dom techniky CSVTS Pardubice 1990.
\end{thebibliography}
