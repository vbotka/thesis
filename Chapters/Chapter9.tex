% Chapter 9
\chapter{Conclusions for the practice and development of the discipline.}\label{Chapter9}
\lhead{Chapter 9. \emph{Conclusions for the practice and development of the discipline}}
%- - - - - - - - - - - - - - - - - - - - - - - - - - - - - - - - - - - - - - - - - - -

In terms of the chosen objectives, a number of insights have been
achieved that have been applied in practice in the control of
technological procedures for the creation of semiconductor structures
by planar technology. In conclusion, the contributions of of the work
can be summarized in the following points:

\begin{enumerate}

% 1
\item Implementation of a complex automated workplace for the
  investigation of electrophysical properties of MOS structures with
  inhomogeneous with the possibility of monitoring the planar
  distribution of impurities:

  \begin{itemize}
  \item concentration profile of the dopant impurity $N(x)$ for
    different depths $x$
  \item of the depth profile of the lifetime $\tau_{g}(x)$ for
    different depths $x$
  \item of the $Si-SiO_{2}$ interface trap density $D_{it}(E_{c}-E)$
    for different energies in the forbidden band of the semiconductor
  \item of the voltages of the aligned strips $V_{fb}$
  \item of the oxide layer thickness $h_{ox}$.
  \end{itemize}

% 2
\item Selection of appropriate numerical methods and their use for the
  solution:

  \begin{itemize}
  \item one-dimensional Poisson equation
  \item nonlinear equation for determining the surface potential from
    OPN capacitance $C_{sc}$
  \item of smoothing and interpolation of experimentally determined data
  \item calculation of the derivative of experimentally determined data.
  \end{itemize}

% 3
\item Creation of software for experimental control measurements,
  processing and displaying the results of selected parameters of MOS
  structures with inhomogeneous distribution of impurities in the
  subsurface area of the semiconductor and their areal distribution.

% 4
\item Investigation of homogeneity of implantation process on 4-inch
  silicon wafers with a dose range from $0.6 \times 10^{14}$ to $60.0
  \times 10^{14} m^{-2}$ with respect to:

  \begin{itemize}
  \item depth profile of active impurities
  \item properties of the $Si-SiO_{2}$ interface
  \item depth profile of the generation lifetime of minority carriers
    of charge.
  \end{itemize}

% 5
\item A methodology for determining the implanted dose of
  impurities. Experimental results were verified by simulation of the
  technological process using Pearson IV function. Comparison of
  experimental and theoretical results show minimal difference. The
  proposed methodology for controlling the implanted dose is
  applicable in practice.

% 6
\item A correlation between the concentration profile was found
  between the concentration profile of dopants and the depth profile
  of the generation lifetime of minority charge carriers. The lifetime
  profile of high quality silicon substrates is not determined by the
  dispersion mechanism on random substrate defects, but only on the
  implanted impurities.

\end{enumerate}
