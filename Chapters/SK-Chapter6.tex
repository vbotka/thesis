% Chapter 6

\chapter{Spracovanie dát a zobrazenie získaných parametrov štruktúr MOS.}% Main chapter title
\label{Chapter6} % For referencing the chapter elsewhere, use \ref{Chapter1} 
\lhead{Kapitola 6. \emph{Spracovanie dát a zobrazenie získaných parametrov štruktúr MOS}}

V tejto kapitole uvedieme stručne postup spracovania dát, nameraných
pomocou programov zberu dát a zameriame sa na plošné zobrazenie
získaných parametrov štruktúr MOS\@. Metodika výpočtu parametrov
štruktúr MOS bola popísaná v kapitolách 3 a 4, na ktoré sa budeme
odvolávať.

Väčšina parametrov je určovaná z nameraných dát pomocou samostatných
programov a ukladaná do dátových súborov, ktorých štruktúra bola
popísaná v časti 5. Plošné rozloženie vypočítaných parametrov štruktúr
MOS možno potom zobraziť na displeji počítača. Pri zobrazovaní na
displeji je k dispozícii 16 farieb, avšak pri zobrazení obrázkov na
tlačiarni bolo kvôli lepšej rozlíšiteľnosti použitých len 6
farieb. Orientácia kremíkovej dosky na obrázkoch je smerom `kazetou
hore', pričom jeden štvorček zobrazenej plochy predstavuje hodnotu
parametra štruktúry MOS\@. Farba štvorčeka závisí od intervalu, do
ktorého daná hodnota spadá. Škála intervalov je zobrazená v pravej
časti obrázku, pričom číslo uvedené pri jednotlivých farbách
predstavuje dolnú hranicu intervalu. Pretože sa farby v škále
intervalov viackrát opakujú, bolo potrebné označiť, v ktorých
intervaloch sa zobrazované hodnoty nachádzajú. To je označené veľkým
písmenom `X' medzi dolnou hranicou intervalu a prislúchajúcou
farbou. Možno ešte poznamenať, že náhodne odlišné výsledky v
niektorých bodoch kremíkovej dosky spôsobia, že sa v škále intervalov
objaví vyznačený interval, ktorý zjavne nesúvisí s ostatnými
hodnotami, čo možno pripísať nefunkčnej štruktúre MOS\@. Treba ešte
upozorniť, že orientácia škály sa môže pri zobrazení rôznych
parametrov meniť.

\section{Určenie koncentračného profilu dotujúcich prímesí, dávky implantácie a napätia vyrovnaných pásov.}\label{sec:6.1}

Určenie koncentračného profilu dotujúcich prímesí vykonáva samostatný
program, ktorý ako vstupné dáta vyžaduje súbory (a) nameraných HF C-V
závislostí a (b) nameraných kapacít oxidovej vrstvy štruktúry
MOS\@. Výstupom programu je dátový súbor obsahujúci hĺbkové priebehy
koncentračných profilov skúmaných štruktúr MOS, ktoré sú počítané
podľa vzťahov uvedených v časti~\ref{sec:4.1}. Ak boli v procese zberu
dát zmerané aj kvázistatické C-V závislosti, program vykoná korekciu
koncentračného profilu na hustotu pascí rozhrania
$Si-SiO_2$. Povrchový potenciál určujeme podľa vzťahov~\ref{eq:4.3}
a~\ref{eq:4.4} a používame ho pre korekciu aproximácie hlbokého
ochudobnenia pri povrchu polovodiča a pre výpočet hĺbky (šírky
OPN). Vedľajším produktom je dátový súbor obsahujúci napätia
vyrovnaných pásov $V_{fb}$, ktoré sa určujú v bode C-V závislosti, pre
ktoré povrchový potenciál $\varphi_s$ mení
znamienko. Obrázok~\ref{fig:6.1} predstavuje plošné rozloženie
koncentrácie v rôznych hĺbkach polovodiča a na obrázku~\ref{fig:6.2}
je znázornené rozloženie $V_{fb}$. Koncentračný profil prímesí bol
vytvorený implantáciou $P^{31}$ s dávkou $4.0 \times 10^{15} m^{-2}$
pri energii $120 keV$. Aktivácia prebiehala počas 40 minút pri teplote
$1050 \degree C$ v atmosfére $N_2$. Stredná hodnota $N(x)$ je spolu s
ďalšími priebehmi pre rôzne implantačné dávky zobrazená na
obrázku~\ref{fig:7.3}.

Pre známe priebehy koncentračných profilov môžeme vypočítať dávku
implantácie podľa vzťahu

\begin{equation}\label{eq:6.1}
  D = \int_{0}^{x_{1}}(N(x) - N_{b}) dx
\end{equation}

,kde predpokladáme, že poznáme priebeh $N(x)$ do vzdialenosti $x_{1}$,
pre ktorú platí

\begin{equation}\label{eq:6.2}
  N(x) = N_{b} \qquad {x \ge x_{1}}
\end{equation}

Dávka $D$ vypočítaná týmto spôsobom predstavuje časť implantovaných
iónov, ktoré sa dostali v priebehu implantácie do polovodiča a boli
aktivované v poimplantačnom tepelnom spracovaní. Na
obrázku~\ref{fig:6.3} je znázornené plošné rozloženie dávky na
testovanej kremíkovej doske určenej zo závislosti $N(x)$ zobrazených
na obrázku~\ref{fig:6.1}.

\begin{figure}[h!]\centering
  \includegraphics{Figures/fig-6-1.eps}% chktex-file 8
  \caption[Rozloženie dotujúcich prímesí $N(x)$ v rôznej hĺbke
    $x$]{Rozloženie dotujúcich prímesí $N(x)$ v rôznej hĺbke $x$ pod
    povrchom polovodiča pre dávku implantácie $4.0 \times
    10^{15}m^{-2}$.}\label{fig:6.1}
\end{figure}

\begin{figure}[h!]\centering
  \includegraphics{Figures/fig-6-2.eps}
  \caption[Plošné rozloženie $V_{fb}$]{Plošné rozloženie $V_{fb}$
    určené pri výpočte $N(x)$ z obrázka~\ref{fig:6.1}.}\label{fig:6.2}
\end{figure}

\begin{figure}[h!]\centering
  \includegraphics{Figures/fig-6-3.eps}
  \caption[Plošné rozloženie nameranej dávky implantácie]{Plošné
    rozloženie nameranej dávky implantácie pre profil $N(x)$ z
    obrázka~\ref{fig:6.1}.}\label{fig:6.3}
\end{figure}

\section{Určenie hrúbky oxidovej vrstvy.}\label{sec:6.2}

Ak poznáme kapacitu oxidovej vrstvy štruktúry MOS a jej plochu, potom
môžeme vypočítať jej hrúbku podľa vzťahu

\begin{equation}\label{eq:6.3}
  h_{ox} = A \frac{\epsilon}{C_{ox}}
\end{equation}

kde sme použili hodnotu relatívnej permitivity $SiO_{2}$
$\epsilon_{r}=3.9$. Plošné rozloženie hrúbky oxidovej vrstvy je potom
zobrazené na obrázku~\ref{fig:6.4}.

\begin{figure}[h!]\centering
  \includegraphics{Figures/fig-6-4.eps}
  \caption[Plošné rozloženie hrúbky hradlového oxidu $h_{ox}$]{Plošné
    rozloženie hrúbky hradlového oxidu $h_{ox}$.}\label{fig:6.4}
\end{figure}

\section{Určenie hustoty pascí rozhrania $Si-SiO_{2}$.}\label{sec:6.3}

Pre výpočet hustoty pascí rozhrania použijeme porovnanie HF a
kvázistatickej C-V závislosti, kde použijeme postup popísaný v
časti~\ref{sec:4.2.1}. Možno poznamenať, že pre určenie polohy Fermiho
hladiny na povrchu polovodiča použijeme hodnoty povrchového potenciálu
$\varphi_{s}(V_{g})$, získané integráciou kvázistatickej C-V
závislosti a hodnotu integračnej konštanty $\varphi_{s0}$ určíme
postupom popísaným v dodatku~\ref{app:AppendixG}.

Celý postup výpočtu $D_{it}$ je realizovaný dvoma programami. Prvý
program na základe zmeraných kvázistatických C-V závislostí a známych
hĺbkových priebehov $N(x)$ vypočíta závislosti $\varphi_{s}(V_{g})$,
ktoré uloží do dátového súboru. Druhý program pre svoju činnosť
vyžaduje dátové súbory

\begin{itemize}
\item HF C-V závislosti $C_{mos}^{HF}(V_{g})$
\item kvázistatické C-V závislosti $C_{mos}^{LF}(V_{g})$
\item závislosť povrchového potenciálu od napätia hradla $\varphi_{s}(V_{g})$.
\end{itemize}

Vypočítané hodnoty $D_{it}$ ako závislosť polohy Fermiho hladiny v
zakázanom pásme polovodiča uloží do dátového súboru. Na
obrázku~\ref{fig:6.5} je zobrazené plošné rozloženie $D_{it}$ v strede
zakázaného pásma.

\begin{figure}[h!]\centering
  \includegraphics{Figures/fig-6-5.eps}
  \caption[Plošné rozloženie hustoty pascí rozhrania $Si-SiO_{2}$ v
    strede zakázaného pásma]{Plošné rozloženie hustoty pascí rozhrania
    $Si-SiO_{2}$ v strede zakázaného pásma.}\label{fig:6.5}
\end{figure}

\section{Určenie generačného času života minoritných nosičov náboja.}\label{sec:6.4}

V procese zberu dát metódou konštantnej šírky OPN bol vytvorený dátový
súbor obsahujúci smernice závislostí $V_{g}(t)$ pre rôzne šírky OPN
testovaných štruktúr MOS kremíkovej dosky. Pre určenie generačného
času života minoritných nosičov náboja podľa vzťahu~\ref{eq:3.10} je
dôležité ako vypočítame deriváciu závislosti smerníc $V_{g}(t)$ podľa
vzdialenosti hranice OPN od povrchu polovodiča. Použitie číslicových
filtrov v tomto prípade nie je vhodné, pretože máme k dispozícii málo
bodov závislosti $dV_{g}/dt = f(w)$. V tomto prípade možno použiť
aproximáciu lokálnymi polynómami. Teoretický základ aj zdrojový text
procedúry v jazyku Algol je uvedený napríklad v~\cite{6.1}. Na
obrázku~\ref{fig:6.6} je zobrazené plošné rozloženie generačnej doby
minoritných nosičov náboja, ktorá predstavuje jej strednú hodnotu v
oblasti od $0.9\mu m$ do $1.3\mu m$.

\begin{figure}[h!]\centering
  \includegraphics{Figures/fig-6-6.eps}
  \caption[Plošné rozloženie generačného času života minoritných
    nosičov náboja]{Plošné rozloženie generačného času života
    minoritných nosičov náboja pre oblasť polovodiča od $0.9 \mu m$ do
    $1.3 \mu m$.}\label{fig:6.6}
\end{figure}

\begin{thebibliography}{}
\bibitem[6.1]{6.1}
  Ludwig R.: Methoden der Fehler und Ausgleichrechnung. VEB Berlin 1969.\ s.103.
\end{thebibliography}
